\documentclass{article}

\usepackage{amsmath,amssymb}

\usepackage[dvipsnames]{xcolor}

\usepackage[framemethod=TikZ]{mdframed}
\mdfdefinestyle{theoremFrame}{%
    linecolor=black,
    outerlinewidth=1pt,
    %roundcorner=20pt,
    innertopmargin=0.5\baselineskip,
    innerbottommargin=0.5\baselineskip,
    rightmargin=-10pt,
    leftmargin=-10pt %, backgroundcolor=gray!50!white
    }

\newtheorem{innercustomgeneric}{\customgenericname}
\providecommand{\customgenericname}{}
\newcommand{\newcustomtheorem}[2]{%
  \newenvironment{#1}[1]
  {%
   \renewcommand\customgenericname{#2}%
   \renewcommand\theinnercustomgeneric{##1}%
   \innercustomgeneric
  }
  {\endinnercustomgeneric}
}

\newcustomtheorem{customthm}{Theorem}
\newcustomtheorem{customlemma}{Lemma}

\begin{document}

\noindent\textsf{\textbf{\textcolor{MidnightBlue}{Jacobs University Bremen \hfill Spring Semester 2022}}}\\

\noindent{}\textsf{\textbf{\textcolor{MidnightBlue}{Dr. D. Sinden}}}\\

\begin{center}
{\Large \textbf{\textsf{\textcolor{MidnightBlue}{CA-MATH-804: Numerical Analysis}}}}
\end{center}

\begin{flushright}
\textbf{\textsf{\textcolor{MidnightBlue}{Assignment Sheet 2. Due: February 23, 2022}}}
\end{flushright}

\begin{center}
\textcolor{MidnightBlue}{\rule[0.5\baselineskip]{1.0cm}{0.4pt}}
\end{center}

% first 
\noindent\textsf{\textbf{Exercise 1 [5 Points]}}: Assuming $p, q = 1, 2, \infty, F$  recover the following table of equivalence constants $c_{pq}$ such that $\forall A \in \mathbb{R}^{n \times n}$ we have ${\left\Vert A \right\Vert_{p} \le c_{pq} \left\Vert A \right\Vert_q}$
\begin{center}
\begin{tabular}{ccccc} \hline
$c_{pq}$   & $q=1$      & $q=2$      & $q=\infty$ & $q=F$ \\ \hline
& \\[\dimexpr-\normalbaselineskip+2pt]
$p=1$\hspace{5pt}       & $1$        & $\sqrt{n}$ & $n$        & $\sqrt{n}$ \\
$p=2$\hspace{5pt}       & $\sqrt{n}$ & $1$        & $\sqrt{n}$ & $1$ \\
$p=\infty$ & $n$        & $\sqrt{n}$ & $1$        & $\sqrt{n}$ \\
$p=F$\hspace{2.18056pt} & $\sqrt{n}$ & $\sqrt{n}$ & $\sqrt{n}$ & $1$ \\ \hline
\end{tabular}
\end{center}

%\newlength{\test}%
%\noindent
%\settowidth{\test}{$p=\infty$}\the\test\\
%\settowidth{\test}{$p=F$}\the\test\\
%\settowidth{\test}{$p=1$}\the\test\\
%\settowidth{\test}{$p=2$}\the\test\\

% second
\vspace{\baselineskip}
\noindent\textsf{\textbf{Exercise 2 [5 Points]}}: For any square matrix ${A \in \mathbb{R}^{n \times n}}$, prove the following relations
\begin{itemize}
\item[\textsf{\textbf{a})}] ${\dfrac{1}{n} K_2 (A) \le K_1 (A) \le n K_2(A)}$,
\item[\textsf{\textbf{b})}] ${\dfrac{1}{n} K_\infty (A) \le K_2 (A) \le n K_{\infty}(A)}$,
\item[\textsf{\textbf{c})}] ${\dfrac{1}{n^2} K_1 (A) \le K_\infty (A) \le n^2 K_{1}(A)}$,
\end{itemize}
\noindent
where $K_p (A) = \left\Vert A^{} \right\Vert_{p} \, \left\Vert A^{-1} \right\Vert_{p}$. These relations show that if a matrix is ill-conditioned in a certain norm, it remains so even in another norm, up to a scaling factor.

% third
\vspace{\baselineskip}
\noindent\textsf{\textbf{Exercise 3 [5 Points]}}: Prove the following claims:
\begin{itemize}
\item[\textsf{\textbf{a})}] If ${A \in \mathbb{R}^{n \times n}}$ fulfils one of the following criteria:
\begin{itemize}
\item[\textsf{1.}] strict row-sum criterion (strict diagonal dominance) ${\sum \limits_{\substack{i=1\\{}i\ne{}j}}^{n} \left| a_{ij} \right| < \left| a_{jj} \right| }$ for all $j$ with ${1 \le j \le n}$.
\item[\textsf{2.}] strict column-sum criterion ${\sum \limits_{\substack{i=1\\{}i\ne{}j}}^{n} \left| a_{ji} \right| < \left| a_{jj} \right| }$ for all $j$ with ${1 \le j \le n}$
\end{itemize}
\noindent
then the Jacobi method converges for any initial guess $x^{(0)}$.
\item[\textsf{\textbf{b})}] Let ${A \in \mathbb{C}^{n \times n}}$, then every eigenvalue $\lambda$ of $A$ fulfils one of the following inequalities
\begin{equation}
\left| \lambda - a_{ii} \right| \le \sum \limits_{\substack{j=1\\{}j\ne{}i}}^{n} \left| a_{ij} \right|. \nonumber
\end{equation}

\end{itemize}

% fourth
\vspace{\baselineskip}
\noindent\textsf{\textbf{Exercise 4 [5 Points]}}: A matrix in which the sum of the absolute values of the entries of a row is equal for every row, is called \emph{row equilibrated}.
\begin{itemize}
\item[\textsf{\textbf{a})}] Show that every regular matrix $A$ can be transformed into a row equilibrated matrix by multiplication with a regular diagonal matrix $D$.
\item[\textsf{\textbf{b})}] Let $A$ and $D$ be as in a). Show that all non-singular diagonal matrices $\tilde{D}$ have
\begin{equation}
K_{\infty} \left( D A \right) \le K_{\infty} \left( \tilde{D} A \right). \nonumber 
\end{equation}
\noindent
\textit{Hint}: Let $C = DA$ and find a lower estimate for the condition number of $\tilde{D} D^{-1}C$ in terms of
the condition number of $C$.
\end{itemize}

% note
%\vspace{0.5\baselineskip}
%\noindent\textit{\small\textcolor{gray!90!white}{Note: This fact leads to the cancellation of significant digits, whenever numbers can be represented with only a finite number of digits (as in floating point arithmetic on computers)}}.

% fifth
\vspace{\baselineskip}
\noindent\textsf{\textbf{Exercise 5 [5 Points]}}: Prove Theorem 10 from class, which is repeated below. \\

\noindent
\begin{mdframed}[style=theoremFrame]

\begin{customthm}{10}%\textbf{Theorem 10} 
For $A,\, \delta A  \in \mathbb{R}^{n \times n}$ and $b,\, \delta b \in \mathbb{R}^n$, consider perturbations of the problem $Ax = b$. Assume there exists ${\gamma > 0}$ such that
\begin{equation}
\left\Vert \delta A \right\Vert \le \gamma \left\Vert A \right\Vert \quad \mbox{and} \quad \left\Vert \delta b \right\Vert \le \gamma \left\Vert b \right\Vert \nonumber
\end{equation}
in suitable norms. Also let $\gamma K(A) < 1$ where $K(A)$ is the condition number of $A$ in the norm used above. Then the perturbation $\delta x$ of the solution fulfils
\begin{equation}
\dfrac{\left\Vert x\delta x \right\Vert}{\left\Vert x \right\Vert} \le \dfrac{1 + \gamma K(A)}{1-\gamma K(A)} \quad \mbox{and} \quad \dfrac{\left\Vert \delta x \right\Vert}{\left\Vert x \right\Vert} \le \dfrac{2 \gamma K(A)}{1-\gamma K(A)}. \nonumber 
\end{equation}
\end{customthm}
\end{mdframed}

\noindent
\textit{Hints}: Remember that ${(A + \delta A)(x + \delta x) = b + \delta b}$. Use compatibility and sub-multiplicativity of the norms. Use Theorem 7 from class which says something about the invertibilty of ${\mathrm{Id} + B}$ for matrices $B$.


\vspace{\baselineskip}
\noindent\textsf{\textbf{Exercise 6 [4 Points]}}: Verify that the matrix $B \in \mathbb{R}^{n\times n}$ defined by 
\begin{equation}
b_{ij} = \left\{ \begin{array}{rl}
1 & \quad\mathrm{if}\,\, i=j, \\
1 & \quad\mathrm{if}\,\, i<j, \\
1 & \quad\mathrm{if}\,\, i>j, 
\end{array} \right. \nonumber
\end{equation}
has determinant $\text{det}\left(B\right)=1$ and that $K_\infty \left(B\right)= n 2^{n-1}$.

\vspace{\baselineskip}
\noindent\textsf{\textbf{Exercise 7 [2 Points]}}: Prove that $K\left( AB \right) \le K\left(A\right) K\left(B\right)$ for any two  matrices $A$, $B \in\mathbb{R}^{n\times n}$.


\end{document}
