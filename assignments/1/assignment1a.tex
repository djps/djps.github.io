\documentclass{article}

\usepackage{amsmath,amssymb}

\usepackage[dvipsnames]{xcolor}

\begin{document}

\noindent\textsf{\textbf{\textcolor{MidnightBlue}{Jacobs University Bremen \hfill Spring Semester 2022}}}\\

\noindent{}\textsf{\textbf{\textcolor{MidnightBlue}{Dr. D. Sinden}}}\\

\begin{center}
{\Large \textbf{\textsf{\textcolor{MidnightBlue}{CA-MATH-804: Numerical Analysis}}}}
\end{center}

\begin{flushright}
\textbf{\textsf{\textcolor{MidnightBlue}{Assignment Sheet 1. Due: February 23, 2022}}}
\end{flushright}

\begin{center}
\textcolor{MidnightBlue}{\rule[0.5\baselineskip]{1.0cm}{0.4pt}}
\end{center}

% first 
\noindent\textsf{\textbf{Exercise 1 [5 Points]}}: Let $X$, $Y$ be some normed spaces with norms ${\left\| \cdot \right\|_X}$ and ${\left\Vert \cdot \right\Vert_Y}$. Furthermore $B : X \rightarrow Y$. Show that the operator norm
\begin{equation}
\left\| B \right\| = \sup_{x \in X}\dfrac{\left\| B x \right\|_Y}{\left\| x\right\|_X}
\nonumber 
\end{equation}
is indeed a norm.

% second
\vspace{\baselineskip}
\noindent\textsf{\textbf{Exercise 2 [5 Points]}}: As in class, let ${ G \left( d \right) = \left( G_{+}(d), G_{-}(d) \right) : \mathbb{R} \rightarrow \mathbb{R}^{2} }$ be the resolvent for solving the quadratic equation
\begin{equation}
x^2 - 2dx +1 = 0, \nonumber
\end{equation}
for $d \ge 1$. Thus, we have $G_{\pm}(d) = \frac{d}{2} \pm \sqrt{d^2 - 1}$. Compute the operator norm $\left\| G^\prime \right\|$ using the Euclidean norm $\left\| \cdot \right\|$.

% third
\vspace{\baselineskip}
\noindent\textsf{\textbf{Exercise 3 [5 Points]}}: Consider Newton's method for the finding of roots of a second-order polynomial
\begin{equation}
x^2 - 2 d x + 1, \nonumber
\end{equation}
for given data $d$. Show that the relative condition number of this numerical method is
\begin{equation}
K_n \left(d\right) = \dfrac{|d|}{|x_{n-1} - d|}. \nonumber 
\end{equation}

% fourth
\vspace{\baselineskip}
\noindent\textsf{\textbf{Exercise 4 [5 Points]}}: Consider the simple linear function $f : \mathbb{R}^2 \rightarrow \mathbb{R}$ with $f(a, b) = a + b$. Use the vector-norm $\left\|x \right\|_1 = \sum_i | x_i |$ on the gradient of $f$ to compute the relative condition number of $f$. To this end make clear what the roles of data $d$ and solution $x$ are and use the formulas we have derived in class. Show that the addition of numbers with equal sign is well conditioned, whereas the subtraction of two numbers, which are almost equal, is ill conditioned. 

% note
\vspace{0.5\baselineskip}
\noindent\textit{\small\textcolor{gray!90!white}{Note: This fact leads to the cancellation of significant digits, whenever numbers can be represented with only a finite number of digits (as in floating point arithmetic on computers)}}.

% fifth
\vspace{\baselineskip}
\noindent\textsf{\textbf{Exercise 5 [5 Points]}}: Let $A \in \mathbb{R}^{n\times n}$ and let ${P\left(A\right) = \sum_{k=0}^{n} c_k A^k}$ be a polynomial in $A$ with coefficients $c_k \in \mathbb{R}$. Check that if $\lambda\left(A\right)$ is the set of eigenvalues of $A$, then the set of eigenvalues of $P\left(A\right)$ is given by $P\left( \lambda\left(A\right)\right)$. 


% sixth
\vspace{\baselineskip}
\noindent\textsf{\textbf{Exercise 6 [5 Points]}}: For a complex number $z=a+ib$ with $a$, $b \in \mathbb{R}$, investigate the stability of the following formula for the square root of $z$
\begin{equation}
\sqrt{z} = \dfrac{1}{\sqrt{2}} \left( \text{sgn}(b) \sqrt{|z|+a} + i\sqrt{|z|-a} \right) \nonumber
\end{equation}

% seventh
\vspace{\baselineskip}
\noindent\textsf{\textbf{Exercise 7 [5 Points]}}: For $a \in \mathbb{R}$ consider the following Cauchy problem:
\begin{align*}
\dot{x}\left( t \right) = x_0 e^{at} \left( a \cos\left(t\right) - \sin\left(t\right) \right) \quad \mbox{for} \quad t \ge 0 \quad \mbox{with} \quad x\left(0\right) = x_0.
\end{align*}
whose is solution is $x\left( t \right) = x_0 e^{at} \cos\left(t\right)$. Use the definition of $K_{\mathrm{abs}}\left(a\right)$ to show that the problem is well conditioned if ${a<0}$ and ill-conditioned if ${a>0}$.
\end{document}




