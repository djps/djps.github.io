\documentclass{article}

\usepackage{amsmath,amssymb}

\usepackage[dvipsnames]{xcolor}

\usepackage[framemethod=TikZ]{mdframed}
\mdfdefinestyle{theoremFrame}{%
    linecolor=black,
    outerlinewidth=1pt,
    %roundcorner=20pt,
    innertopmargin=0.5\baselineskip,
    innerbottommargin=0.5\baselineskip,
    rightmargin=-10pt,
    leftmargin=-10pt %, backgroundcolor=gray!50!white
    }

\newtheorem{innercustomgeneric}{\customgenericname}
\providecommand{\customgenericname}{}
\newcommand{\newcustomtheorem}[2]{%
  \newenvironment{#1}[1]
  {%
   \renewcommand\customgenericname{#2}%
   \renewcommand\theinnercustomgeneric{##1}%
   \innercustomgeneric
  }
  {\endinnercustomgeneric}
}

\newcustomtheorem{customthm}{Theorem}
\newcustomtheorem{customlemma}{Lemma}

\begin{document}

\noindent\textsf{\textbf{\textcolor{MidnightBlue}{Jacobs University Bremen \hfill Spring Semester 2022}}}\\

\noindent{}\textsf{\textbf{\textcolor{MidnightBlue}{Dr. D. Sinden}}}\\

\begin{center}
{\Large \textbf{\textsf{\textcolor{MidnightBlue}{CA-MATH-804: Numerical Analysis}}}}
\end{center}

\begin{flushright}
\textbf{\textsf{\textcolor{MidnightBlue}{Assignment Sheet 4. Due: February 23, 2022}}}
\end{flushright}

\begin{center}
\textcolor{MidnightBlue}{\rule[0.5\baselineskip]{1.0cm}{0.4pt}}
\end{center}

\noindent\textsf{\textbf{Exercise 1 [5 Points]}}: Let $I_n\left( f\right) = \sum_{k=0}^{n} \alpha_k f\left( x_k \right)$ be a Lagrange quadrature formula on $n+1$ nodes. Compute the degree of exactness $r$ for the formula
\begin{equation}
I_4 \left( f\right) = \dfrac{1}{4} \left[ f\left(-1\right) + 3f\left( -\dfrac{1}{3} \right) + 3f\left( \dfrac{1}{3} \right) + f\left(1\right) \right]. \nonumber
\end{equation}


\vspace{\baselineskip}
\noindent\textsf{\textbf{Exercise 2 [5 Points]}}: Let $I_w (f) = \int_0^1 w(x)f(x) \,\mathrm{d}x$ with $w(x) = \sqrt{x}$ and let $Q(f) = \alpha f\left( x_1 \right)$ be a quadrature formula approximating $I_w$. Find $\alpha$ and $x_1$ such that $Q$ has maximum degree of exactness $r$.

\vspace{\baselineskip}
\noindent\textsf{\textbf{Exercise 3 [5 Points]}}: Prove that $G^k\left(x_j\right) = h G\left( x_j, x_k \right)$, where $G$ is the Green’s function for the problem
\begin{equation}
u^{\prime\prime} = f \quad \mbox{in} \left(0,1\right), \\ \nonumber
u\left(x\right) = 0 \quad \mbox{on} \left\{0,1\right\},  \nonumber
\end{equation}
and $G^k$ its corresponding discrete counterpart.



\vspace{\baselineskip}
\noindent\textsf{\textbf{Exercise 4 [3+3 Points]}}: Consider the matrix $A_{\mathrm{fd}} = h^{-2}\mathrm{tridiag}(-1, 2, -1)$ which appears in the finite difference discretization of the second derivative.
\begin{itemize}
\item[\textsf{\textbf{a})}] Show that $A_{\mathrm{fd}}$ is symmetric and positive definite
\item[\textsf{\textbf{b})}] Show that $A_{\mathrm{fd}}$ is an $M$-Matrix, i.e. ${a_{ij} \le 0}$ for ${i \ne j}$ and all the entries of its inverse are nonnegative.
\end{itemize} 


\vspace{\baselineskip}
\noindent\textsf{\textbf{Exercise 5 [3+3 Points]}}: Consider an equidistant grid with nodes $x_i$ and grid-width $h$ and a real
valued function $f$ with sufficient smoothness. Using Taylor series expansion show that
\begin{itemize}
\item[\textsf{\textbf{a})}] $\left| f^{\prime} \left( x_i \right) - D_{i}^{-} f \left( x_i \right) \right| = \dfrac{h}{2} \left| f^{\prime\prime} \left( \xi \right) \right|$ for some ${\xi \in \left(x_{i-1},x_i \right)}$,
\item[\textsf{\textbf{b})}]  $\left| f^{\prime\prime} \left( x_i \right) - D_{i}^{\pm} f \left( x_i \right) \right| = \dfrac{h^2}{24} \left| f^{\prime\prime\prime\prime} \left( \xi_1 \right) + f^{\prime\prime\prime\prime} \left( \xi_2 \right)\right|$ for some ${\xi_1 \in \left(x_{i-1},x_i \right)}$, ${\xi_2 \in \left(x_i,x_{i+1} \right)}$.
\end{itemize} 

\vspace{\baselineskip}
\noindent\textsf{\textbf{Exercise 6 [4 Points]}}: Let $E_0\left(f\right)$ and $E_1\left(f\right)$ be the quadrature errors of the midpoint ans the trapezoidal formula respectively. Prove that ${E_1\left(f\right) \approx 2 \left\| E_0\left(f\right) \right\|}$.


\vspace{\baselineskip}
\noindent\textsf{\textbf{Exercise 7 [4 Points]}}: An alternative approach for the construction of the Lagrange interpolation polynomial $\Pi_n$ involves directly enforcing the interpolation constraints on $\Pi_n$ and then computing the coefficients $a_i$. This produces a system of simultaneous linear equations which can be written as the linear system ${X a=y}$, where the coefficients of the matrix $X$ are given by $X_{ij}=x_i^{j-1}$ where the points $x_i$ are the interpolation points with $i=0, \ldots n$. Prove that the matrix $X$ is invertible if the nodes are distinct.
\textit{Hint}: The matrix $X$ is called the square Vandermonde matrix.

\end{document}