
\documentclass{article}

\usepackage{amsmath,amssymb}

\usepackage[dvipsnames]{xcolor}

\usepackage{geometry}
 \geometry{
 a4paper,
 total={170mm,240mm},
 left=24mm,
 top=24mm,
 }

\usepackage[framemethod=TikZ]{mdframed}
\mdfdefinestyle{theoremFrame}{%
    linecolor=black,
    outerlinewidth=1pt,
    %roundcorner=20pt,
    innertopmargin=0.5\baselineskip,
    innerbottommargin=0.5\baselineskip,
    rightmargin=-10pt,
    leftmargin=-10pt %, backgroundcolor=gray!50!white
    }

\newtheorem{innercustomgeneric}{\customgenericname}
\providecommand{\customgenericname}{}
\newcommand{\newcustomtheorem}[2]{%
  \newenvironment{#1}[1]
  {%
   \renewcommand\customgenericname{#2}%
   \renewcommand\theinnercustomgeneric{##1}%
   \innercustomgeneric
  }
  {\endinnercustomgeneric}
}

\newcustomtheorem{customthm}{Theorem}
\newcustomtheorem{customlemma}{Lemma}

\begin{document}

\noindent\textsf{\textbf{\textcolor{MidnightBlue}{Jacobs University Bremen \hfill Spring Semester 2022}}}\\

\noindent{}\textsf{\textbf{\textcolor{MidnightBlue}{Dr. D. Sinden}}}\\

\begin{center}
{\Large \textbf{\textsf{\textcolor{MidnightBlue}{CA-MATH-804: Numerical Analysis}}}}
\end{center}

\begin{flushright}
\textbf{\textsf{\textcolor{MidnightBlue}{Assignment Sheet 6. Due: }}}
\end{flushright}

\begin{center}
\textcolor{MidnightBlue}{\rule[0.5\baselineskip]{1.0cm}{0.4pt}}
\end{center}

\noindent\textsf{\textbf{Exercise 1 [5 x 4* Points]}}: Consider the bending of a clamped beam subject to a transversal force
$f$, which is described by the boundary value problem
\begin{align*}
u^{\prime\prime\prime\prime}\left(x\right) & = f\left(x\right) \quad \mbox{in} \quad(0,1), \\
u(0) = u(1) & = 0, \\
u^{\prime}(0) = u^{\prime}(1) & = 0.
\end{align*}

\begin{itemize}
\item[\textsf{\textbf{a})}] Show that under certain conditions this problem is equivalent to the following variational (weak)
problem:
\begin{equation}
\left( u^{\prime\prime}, v^{\prime\prime} \right) =  \left( f, v \right) \quad \forall v \in W \nonumber
\end{equation}
where
\begin{equation*}
\begin{gathered}
W = \left\{ v : \left(0,1\right) \rightarrow \mathbb{R} \, \| \, v \, \mbox{and} \, v^\prime \, \mbox{are continuous,} \, v^{\prime\prime} \, \mbox{is piecewise continuous} \right. \nonumber \\
\hspace{1cm} \left.  \mbox{and} \, v(0)=v(1) = v^\prime(0)=v^\prime(1) =0 \right\}. \nonumber
\end{gathered}
\end{equation*}

\item[\textsf{\textbf{b})}] For an interval $I = [a, b]$ define
\begin{equation}
\begin{gathered}
P_3\left(I\right) =\left\{ v : I \rightarrow \mathbb{R} \, \| \, v \, \mbox{is a polynomial of degree $\le$ 3, i.e.} \right. \\
\hspace{1cm} \left. v(x) = a_3 x^3 + a_2 x^2 + a_1 x + a_0 \, \mbox{for} \, a_i \in \mathbb{R} \right\}. \nonumber
\end{gathered}
\end{equation}
Show that $v \in P_3(I)$ is uniquely defined by the values $v(a)$, $v^\prime(a)$, $v(b)$, $v^\prime(b)$ and determine the
corresponding basis functions $b_i(x)$ such that
\begin{equation}
v\left(x\right) = v(a)b_0\left(x\right) + v^\prime(a)b_1\left(x\right) + v(b)b_2\left(x\right) + v^\prime(b) b_3\left(x\right). \nonumber
\end{equation}

\item[\textsf{\textbf{c})}] Starting from b) use a uniform partitioning of $(0, 1)$ to construct a finite dimensional subspace $W_h$ of $W$ consisting of piecewise cubic functions. Specify suitable parameters to describe the functions in $W_h$ and determine the corresponding basis functions of $W_h$. What is the dimension of the resulting finite element space $W_h$?

\item[\textsf{\textbf{d})}] Formulate a finite element method for the problem based on the space $W_h$. Find the corresponding system of equations.

\item[\textsf{\textbf{e})}] Determine the finite element solution in the case of two intervals and ${f = 1}$. Compare with the exact solution.

\end{itemize}

\end{document}
