\documentclass{article}

\usepackage{amsmath,amssymb}

\usepackage[dvipsnames]{xcolor}

\usepackage[framemethod=TikZ]{mdframed}
\mdfdefinestyle{theoremFrame}{%
    linecolor=black,
    outerlinewidth=1pt,
    %roundcorner=20pt,
    innertopmargin=0.5\baselineskip,
    innerbottommargin=0.5\baselineskip,
    rightmargin=-10pt,
    leftmargin=-10pt %, backgroundcolor=gray!50!white
    }

\newtheorem{innercustomgeneric}{\customgenericname}
\providecommand{\customgenericname}{}
\newcommand{\newcustomtheorem}[2]{%
  \newenvironment{#1}[1]
  {%
   \renewcommand\customgenericname{#2}%
   \renewcommand\theinnercustomgeneric{##1}%
   \innercustomgeneric
  }
  {\endinnercustomgeneric}
}

\newcustomtheorem{customthm}{Theorem}
\newcustomtheorem{customlemma}{Lemma}

\begin{document}

\noindent\textsf{\textbf{\textcolor{MidnightBlue}{Jacobs University Bremen \hfill Spring Semester 2022}}}\\

\noindent{}\textsf{\textbf{\textcolor{MidnightBlue}{Dr. D. Sinden}}}\\

\begin{center}
{\Large \textbf{\textsf{\textcolor{MidnightBlue}{CA-MATH-804: Numerical Analysis}}}}
\end{center}

\begin{flushright}
\textbf{\textsf{\textcolor{MidnightBlue}{Assignment Sheet 3. Due: February 23, 2022}}}
\end{flushright}

\begin{center}
\textcolor{MidnightBlue}{\rule[0.5\baselineskip]{1.0cm}{0.4pt}}
\end{center}

\noindent\textsf{\textbf{Exercise 1 [5 Points]}}: Consider the matrix
\begin{equation}
A = \left( \begin{array}{cc}
1 & \gamma \\
0 & 1
\end{array}
\right) . \nonumber
\end{equation}
\begin{itemize}
\item[\textsf{\textbf{a})}] Show that for ${\gamma \le 0}$ we have ${K_{\infty}(A) = K_1(A) = \left(1+\gamma\right)^{2}}$. 
\item[\textsf{\textbf{b})}] For the linear system $Ax = b$, where $b$ is such that $x=\left( 1 - \gamma, 1 \right)^T$ is the solution, find a bound for $\left\Vert \delta x \right\Vert_{\infty}\slash \left\Vert x \right\Vert_{\infty}$ in terms of $\left\Vert \delta b \right\Vert_{\infty}\slash \left\Vert b \right\Vert_{\infty}$ when $\delta b = \left( \delta_1, \delta_2 \right)^T$. What can you say about the condition of the problem?
\end{itemize}
\noindent

\vspace{\baselineskip}
\noindent\textsf{\textbf{Exercise 2 [5 Points]}}: Check that the matrix $A = \mathrm{tridiag}_n \left( -1,\alpha, -1 \right)$ with $\alpha \in \mathbb{R}$ has eigenvalues
given by
\begin{equation}
\lambda_i = \alpha - 2 \cos( i \theta), \quad i= 1,\ldots,n, \nonumber
\end{equation}
and the corresponding eigenvectors are
\begin{equation}
v_i = \left( \sin\left(i\theta\right), \sin\left(2i\theta\right), \ldots, \sin\left(ni\theta\right)\right)^T , \nonumber
\end{equation}
where $\theta = \dfrac{\pi}{n+1}$.

\vspace{\baselineskip}
\noindent\textsf{\textbf{Exercise 3 [5 Points]}}: Consider the linear system $Ax=b$, where
\begin{equation}
A = \left( \begin{array}{ccc}
1 & 1 & 1 \\
\alpha & 1 & 1 \\ 
\beta & \gamma & 1\\
\end{array}
\right) \nonumber
\end{equation}
and $\alpha, \beta, \gamma \in \mathbb{R}$. Define an iterative method for ${k \ge 0}$ as
\begin{equation}
x^{\left(k+1\right)} = U^{-1}\left( b - L x^{\left(k\right)} \right), \nonumber
\end{equation}
where
\begin{equation}
U = \left( \begin{array}{ccc}
1 & 1 & 1 \\
0 & 1 & 1 \\ 
0 & 0 & 1\\
\end{array}
\right) \quad \mbox{and} \quad L = \left( \begin{array}{ccc}
0 & 0 & 0 \\
\alpha & 0 & 0 \\ 
\beta & \gamma & 0\\
\end{array}
\right). \nonumber
\end{equation}
\begin{itemize}
\item[\textsf{\textbf{a})}] Find all values of $\alpha, \beta, \gamma$ such that the sequence of iterates $\left\{ x^{(k)} \right\}$ converges for every initial guess $x^{(0)}$ and every right hand side.
\item[\textsf{\textbf{b})}] Give an example for $b$ leading to non-convergence in the case ${\alpha = \beta = \gamma = -1}$.
\item[\textsf{\textbf{c})}] Is the solution always found with at most two iterations if ${\alpha = \gamma = 0}$?
\end{itemize} 

\vspace{\baselineskip}
\noindent\textsf{\textbf{Exercise 4 [3 $\boldsymbol{\times}$ 2 Points]}}: Compute the following derivatives:
\begin{itemize}
\item[\textsf{\textbf{a})}] $\dfrac{\mathrm{d}}{\mathrm{d}s} \left\Vert x + sp \right\Vert_q^2$ for $x,p \in\mathbb{R}$ and $1\le q < \infty$,
\item[\textsf{\textbf{b})}] $\nabla \left\Vert u(x) \right\Vert_2^2$ where $u : \mathbb{R}^n \rightarrow \mathbb{R}^n$ is sufficiently smooth,
\item[\textsf{\textbf{c})}] $\Delta \left\Vert x \right\Vert_2^2 : = \mathrm{div} \left( \left\Vert x \right\Vert_2^2\right) := \nabla \cdot \left( \nabla \left\Vert x \right\Vert_2^2 \right) $ for $x \in \mathbb{R}^n$.
\end{itemize} 

\vspace{\baselineskip}
\noindent\textsf{\textbf{Exercise 5 [5 Points]}}: Given $n + 1$ distinct points $x_0, \ldots, x_n \in \mathbb{R}$ consider the functions
\begin{equation}
l_i = \prod \limits_{\substack{j=0\\{}j\ne{}i}}^{n} \dfrac{x-x_j}{x_i - x_j}, \quad i = 0, \ldots n. \nonumber
\end{equation}
Show that these functions form a basis for the polynomials $\mathbb{P}_n$ of degree $\le n$ over $\mathbb{R}$.

\vspace{\baselineskip}
\noindent\textsf{\textbf{Exercise 6 [6 Points]}}: Consider the linear system $Ax=b$ with
\begin{equation}
A = \left( \begin{array}{cc}
1 & 2 \\
3 & 4
\end{array} \right), \quad b =  \left( \begin{array}{c}
3 \\
5
\end{array} \right), \nonumber
\end{equation}
and the following iterative method
\begin{equation}
x^{\left(k+1\right)} = B\left(\theta\right) x^{(k)} + g(\theta) \nonumber
\end{equation}
for $k \ge 0$ and a given $x^{(0)}$, where $\theta \in \mathbb{R}$ and
\begin{equation}
B\left(\theta\right) = \dfrac{1}{4}\left( \begin{array}{cc}
2 \theta^2 + 2\theta +1 & -2 \theta^2 + 2\theta +1 \\
-2 \theta^2 + 2\theta +1 & 2 \theta^2 + 2\theta +1
\end{array} \right), \quad g(\theta) =  \left( \begin{array}{c}
\frac{1}{2} - \theta \\
\frac{1}{2} - \theta 
\end{array} \right). \nonumber
\end{equation}
\begin{itemize}
\item[\textsf{\textbf{a})}] Show that the method is consistent for all $\theta \in \mathbb{R}$.
\item[\textsf{\textbf{b})}] Show that the method is convergent if and only if ${-1 < \theta < \frac{1}{2}}$.
\item[\textsf{\textbf{c})}] Show that the rate of convergence is maximal for the value ${\theta = \frac{1}{2}\left( 1- \sqrt{3} \right) }$.
\end{itemize} 


\end{document}