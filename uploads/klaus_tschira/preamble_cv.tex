% Configure page margins with geometry
\geometry{left=1.4cm, top=0.8cm, right=1.4cm, bottom=1.8cm, footskip=.5cm}

% Specify the location of the included fonts
%\fontdir[/usr/share/texlive/texmf-dist/fonts/]

\usepackage{fontspec}

% Set Raleway as the main font
\setmainfont{Raleway}

% Color for highlights
% Awesome Colors: awesome-emerald, awesome-skyblue, awesome-red, awesome-pink, awesome-orange, awesome-nephritis, awesome-concrete, awesome-darknight

% this is teal
\definecolor{awesome}{HTML}{008080}

% tailwind-cyan-500
\definecolor{awesome}{HTML}{06b6d4}
% tailwind-cyan-600
\definecolor{awesome}{HTML}{0891b2} 
\definecolor{awesome}{RGB}{8, 145, 178}

\definecolor{awesome-faded}{RGB}{85, 222, 255}



% Colors for text
% Uncomment if you would like to specify your own color
% \definecolor{darktext}{HTML}{414141}
% \definecolor{text}{HTML}{333333}
% \definecolor{graytext}{HTML}{5D5D5D}
% \definecolor{lighttext}{HTML}{999999}

% Set false if you don't want to highlight section with awesome color
\setbool{acvSectionColorHighlight}{false}

% This overwrites the boolean above
\makeatletter
\def\@sectioncolor{\color{awesome}}
\makeatother

% If you would like to change the social information separator from a pipe (|) to something else
\renewcommand{\acvHeaderSocialSep}{\quad\textbar\quad}

\def\endfirstpage{\newpage}







\usepackage{booktabs}
\usepackage{csquotes}

\newlength{\xbigstrutjot}
\setlength{\xbigstrutjot}{1em}

\makeatletter
\@ifundefined{xbigstrutjot}{\newdimen\xbigstrutjot}{} 
\xbigstrutjot=10pt

\newcommand\xbigstrut[1][x]{%
  \leavevmode\unskip\@tempdima=\ht\@arstrutbox \@tempdimb=\dp\@arstrutbox
  \ifx #1b\relax \else \advance\@tempdima by \xbigstrutjot\fi
  \ifx #1t\relax \else \advance\@tempdimb by \xbigstrutjot\fi
  \hbox{\textcolor{blue}{\vrule} \@height\@tempdima \@depth\@tempdimb \@width\z@}\ignorespaces}
\makeatother

 \newcommand{\customstrut}{\textcolor{blue}{\rule{0pt}{0.5em}}}

\usepackage[calc]{datetime2}

\DTMsavenow{mytoday}

\providecommand{\tightlist}{\setlength{\itemsep}{0pt}\setlength{\parskip}{0pt}}

%------------------------------------------------------------------------------

% Pandoc CSL macros
\newlength{\cslhangindent}
\setlength{\cslhangindent}{1.5em}
\newlength{\csllabelwidth}
\setlength{\csllabelwidth}{3em}
\newenvironment{CSLReferences}[3] % #1 hanging-ident, #2 entry spacing
 {% don't indent paragraphs
  \setlength{\parindent}{0pt}
  % turn on hanging indent if param 1 is 1
  \ifodd #1 \everypar{\setlength{\hangindent}{\cslhangindent}}\ignorespaces\fi
  % set entry spacing
  \ifnum #2 > 0
  \setlength{\parskip}{#2\baselineskip}
  \fi
 }%
 {}
 
\usepackage{calc}
\newcommand{\CSLBlock}[1]{#1\hfill\break}
\newcommand{\CSLLeftMargin}[1]{\parbox[t]{\csllabelwidth}{#1}}
\newcommand{\CSLRightInline}[1]{\parbox[t]{\linewidth - \csllabelwidth}{#1}}
\newcommand{\CSLIndent}[1]{\hspace{\cslhangindent}#1}

\usepackage{colortbl}

\usepackage{printlen}

%%%%%%%%%%%%%%%%
\RequirePackage{tikz}
\usepackage{adjustbox}
\usepackage{mdframed}

\usepackage{verbatimbox}

\RequirePackage{multirow}
\RequirePackage{arydshln} % incompatible with fancy style

\makeatletter

\newcommand*{\@initializelength}[1]{%
  \ifdefined#1
  \else%
    \newlength{#1}\fi%
  \setlength{#1}{0pt}}

% creates an if switch if not yet defined
\newcommand*{\@initializeif}[1]{%
%  \ifdefined#1% not working due to the nested \if
%  \else%
    \newif#1%\fi
  }

\@initializeif{\if@aftersection}\@aftersectionfalse%

% definitions form elsewhere
\newcommand*{\hintfont}{}
\newcommand*{\hintstyle}[1]{{\hintfont#1}}

\renewcommand*{\hintstyle}[1]{{\hintfont\textcolor{awesome}{#1}}}

\@initializelength{\separatorcolumnwidth}         
\setlength{\separatorcolumnwidth}{0.025\textwidth}

\@initializelength{\hintscolumnwidth}             
\setlength{\hintscolumnwidth}{0.0\textwidth}

\@initializelength{\maincolumnwidth}
\setlength{\maincolumnwidth}{\textwidth-\leftskip-\rightskip}%

\@initializelength{\cvSkill@RectangleSize}
\setlength{\cvSkill@RectangleSize}{1.2ex}
\newcount\my@repeat@count

\DeclareDocumentCommand{\cvskill}{m}{%
%   Illustrate skill level with little colored boxes.
%   By default five skill levels are predefined.   
% 
%   Input: 
%       integer between 0 and 5
%   Example usage: 
%       \cvskill{3}
\smash{\raisebox{-0.65\baselineskip}{
%\fbox{
    \begingroup
        \my@repeat@count=\z@
        \@whilenum\my@repeat@count<#1\do{\tikz\filldraw[awesome] (0, 0) rectangle (\cvSkill@RectangleSize, \cvSkill@RectangleSize);\advance%
        \my@repeat@count\@ne\,}%
        \my@repeat@count=\numexpr5-\z@\relax
        \@whilenum\my@repeat@count>#1\do{\tikz\filldraw[awesome-faded] (0, 0) rectangle (\cvSkill@RectangleSize, \cvSkill@RectangleSize);\advance%
        \my@repeat@count\m@ne\,}%
    \endgroup
    %}
    }}
}% end \cvskill

\@initializelength{\cvskill@width}
\@initializelength{\cvskill@descriptorwidth}
\@initializelength{\cvskill@experiencewidth}

\@initializelength{\skillmatrix@columnwidth}
\@initializelength{\skillmatrix@commentwidth}
\@initializelength{\skillmatrix@padding}
\@initializelength{\skillmatrix@hintscolumnwidth}
\@initializelength{\skillmatrix@bodylength}

\@initializelength{\cvskilllegend@leftdescriptorwidth}
\@initializelength{\cvskilllegend@rightdescriptorwidth}

\@initializelength{\skilllegend@hintscolumnwidth}
\@initializelength{\skilllegend@padding}
\@initializelength{\skilllegend@bodylength}

\@initializelength{\separatorrulewidth}

\DeclareDocumentCommand{\skilllegend@leftdesriptorfactor}{}{}%

%% DEFINITION \recompute@cvskillmatrix@lengths
% declare the command \recompute@cvskillmatrix@lengths empty

\DeclareDocumentCommand{\recompute@cvskillmatrix@lengths}{}{}%

\RenewDocumentCommand{\recompute@cvskillmatrix@lengths}{}{%

\setlength{\skillmatrix@padding}{1ex}%

\setlength{\skillmatrix@hintscolumnwidth}{\hintscolumnwidth}%

\setlength{\cvskill@width}{\widthof{\cvskill{5}}}%

\setlength{\cvskill@experiencewidth}{\widthof{``Year''}}%

\setlength{\skillmatrix@bodylength}{\maincolumnwidth}%

\setlength{\skillmatrix@columnwidth}{0.45\skillmatrix@bodylength}%

\setlength{\cvskill@descriptorwidth}{\skillmatrix@columnwidth-\cvskill@width-\cvskill@experiencewidth}%

\setlength{\skillmatrix@commentwidth}{\skillmatrix@bodylength-\skillmatrix@columnwidth-\skillmatrix@padding}%

% lengths and definitions needed for the legends
% note that \skillmatrix@padding also affects the appearance of legends
\setlength{\skilllegend@padding}{0.25ex}%

\setlength{\skilllegend@hintscolumnwidth}{\hintscolumnwidth}%

\setlength{\skilllegend@bodylength}{\skillmatrix@bodylength}%

\RenewDocumentCommand{\skilllegend@leftdesriptorfactor}{}{0.5}%

%\setlength{\cvskill@descriptorwidth}{5cm}%
%\setlength{\skillmatrix@commentwidth}{10cm}%
%\setlength{\skilllegendbodylength}{\skillmatrix@bodylength}%
}

% \setcvskillcolumns[<width>][<factor>][<exp_width>]
\DeclareDocumentCommand{\setcvskillcolumns}{+O{\skillmatrix@hintscolumnwidth} +O{\skillmatrix@columnwidth} +O{\cvskill@experiencewidth}}{%
%   adjust column width of skill table
% 
%   The \cvskillentry command comes with default FIXED width definitions of the kill matrix for the columns 
%   based on the author's own skill matrix. This is necessary to ensure that the entries are nicely aligned 
%   and actually yield a decent looking table. The defaults depend on the style used and are chosen reasonably.
%   However, depending on the user input and the style that is used some of the columns might need adjustments. 
%   The \setcvskillcolumns command provides means to influence the width of the first, the third and the fourth 
%   skill matrix column. The second column containg the output of \cvskill remains fixed width. The last column,
%   the comment column gets recalculated according to the setting of the other columns.
% 
%   Input
%       Input_1 (optional):     width smaller than \textwidth,  default <\skillmatrix@hintscolumnwidth>
%       Input_2 (optional):     float between 0 and 1 adjusting how much percent of the table width without 
%                               the first column is used columns two, three and four. Through this parameter
%                               the width of the 3rd column (skill name) can be adjusted,   default <\skillmatrix@columnwidth>
%       Input_3 (optional):     width smaller than \textwidth, setting the width of the 4th column 
%                               (Years of experience), default <\cvskill@experiencewidth>
% 
%   Example usage
%       \setcvskillcolumns[5em][][]%    adjust first column. Same as \setcvskillcolumns[5em]
%       \setcvskillcolumns[][0.45][]%   adjust third (skill) column. Same as \setcvskillcolumns[][0.45]
%       \setcvskillcolumns[][][\widthof{``Year''}]%     adjust fourth (years) column.
%       \setcvskillcolumns[\widthof{``Language''}][0.48][]%     adjust 1st and 3rd columns. Same as \setcvskillcolumns[\widthof{``Language''}][0.45]    
%       \setcvskillcolumns[\widthof{``Management Tools''}][0.6][3em]%   ajust all at once.
% 
%   Note
%       - For the styles 'classic' and 'casual' the first column is set to hintscolumnwidth such that
%         it aligns with the rest of the entries. A readjustment of the first column should therefor 
%         be avoided. It is recomended to only use \setcvskillcolumns in the form of 
%         \setcvskillcolumns[][<factor>][<width>], thereby leaving the defaults in place for the first column.
% 
    \def\arg@new@hintscolumnwidth{#1}% <-- all these terminal % signs are necessary for the fancy style to not show weird spaces!!!
    \def\arg@new@bodyLengthFactor{#2}% 
    \def\arg@new@experienceWidth{#3}% 
    % Check for empty arguments. Defaults are given. Thus a call of \setcvskillcolumns 
    % without any arguments leads to nonempty arguments \arg@new@hintscolumnwidth and
    % \def\arg@new@bodyLengthFactor{#2}. However, we need to take care of calls like 
    % \setcvskillcolumns[], \setcvskillcolumns[][], \setcvskillcolumns[][][] or even 
    % \setcvskillcolumns[<somelength>][], \setcvskillcolumns[][<somefactor>]  \setcvskillcolumns[][][<length>]
    \ifdefempty{\arg@new@hintscolumnwidth}{%
        % Case \setcvskillcolumns[], \setcvskillcolumns[][] or \setcvskillcolumns[][<somefactor>]
        \ifdefempty{\arg@new@bodyLengthFactor}{%
            % Case \setcvskillcolumns[][] do nothing here and check if third argument is empty
            \ifdefempty{\arg@new@experienceWidth}{%
                % Case \setcvskillcolumns[][][] do nothing here
            }{%
                % Case \setcvskillcolumns[][][<length>]. reset \cvskill@experiencewidth and
                % \cvskill@descriptorwidth accordingly
                \setlength{\cvskill@experiencewidth}{\arg@new@experienceWidth}%
                \setlength{\cvskill@descriptorwidth}{\skillmatrix@columnwidth-\cvskill@width-\cvskill@experiencewidth}%
            }%
        }{%
            % Case \setcvskillcolumns[][<somefactor>], \setcvskillcolumns[][<somefactor>][<possilly length>]
            \setlength{\skillmatrix@columnwidth}{\arg@new@bodyLengthFactor\skillmatrix@bodylength}%
            \ifdefempty{\arg@new@experienceWidth}{%
                % Case \setcvskillcolumns[][<somefactor>][] do nothing here
            }{%
                % Case \setcvskillcolumns[][<somefactor>][<length>]. reset \cvskill@experiencewidth and
                % \cvskill@descriptorwidth accordingly
                \setlength{\cvskill@experiencewidth}{\arg@new@experienceWidth}%
                \setlength{\cvskill@descriptorwidth}{\skillmatrix@columnwidth-\cvskill@width-\cvskill@experiencewidth}%
            }%            
            \setlength{\cvskill@descriptorwidth}{\skillmatrix@columnwidth-\cvskill@width-\cvskill@experiencewidth}%
            \setlength{\skillmatrix@commentwidth}{\skillmatrix@bodylength-\skillmatrix@columnwidth-3\skillmatrix@padding}%
        }%
        % Case \setcvskillcolumns[] nothing needs to be done here recalculate lengths affected by the change
    }{% 
        % Case \setcvskillcolumns, \setcvskillcolumns[<width>], \setcvskillcolumns[<width>][] 
        % or \setcvskillcolumns[<width>][<somefactor>]
        \setlength{\skillmatrix@hintscolumnwidth}{\arg@new@hintscolumnwidth}%
        \setlength{\skillmatrix@bodylength}{\maincolumnwidth-\skillmatrix@hintscolumnwidth-\separatorcolumnwidth}%
        % in case second argument is given but left empty use default
        \ifdefempty{\arg@new@bodyLengthFactor}{%
            % Case \setcvskillcolumns[<width>][] do nothing here and use default
            % \skillmatrix@columnwidth and check third argument
            \ifdefempty{\arg@new@experienceWidth}{%
                % Case \setcvskillcolumns[<width>][][] do nothing here
            }{%
                % Case \setcvskillcolumns[<width>][][<length>]. reset \cvskill@experiencewidth and
                % \cvskill@descriptorwidth accordingly
                \setlength{\cvskill@experiencewidth}{\arg@new@experienceWidth}%
%                 \setlength{\cvskill@descriptorwidth}{\skillmatrix@columnwidth-\cvskill@width-\cvskill@experiencewidth}%
            }%
        }{%
            % Case \setcvskillcolumns, \setcvskillcolumns[<width>], \setcvskillcolumns[<width>][<somefactor>]
            \setlength{\skillmatrix@columnwidth}{\arg@new@bodyLengthFactor\skillmatrix@bodylength}%
            \ifdefempty{\arg@new@experienceWidth}{%
                % Case \setcvskillcolumns[<width>][<somefactor>][] do nothing here
            }{%
                % Case \setcvskillcolumns[<width>][<somefactor>][<length>]. reset \cvskill@experiencewidth and
                % \cvskill@descriptorwidth accordingly
                \setlength{\cvskill@experiencewidth}{\arg@new@experienceWidth}%
%                 \setlength{\cvskill@descriptorwidth}{\skillmatrix@columnwidth-\cvskill@width-\cvskill@experiencewidth}%
            }%
        }%
        \setlength{\cvskill@descriptorwidth}{\skillmatrix@columnwidth-\cvskill@width-\cvskill@experiencewidth}%
        \setlength{\skillmatrix@commentwidth}{\skillmatrix@bodylength-\skillmatrix@columnwidth-3\skillmatrix@padding}%
    }%
}%


% %-------------------------------------------------------------------------------
% %                \cvskillentry 
% %-------------------------------------------------------------------------------
% \cvskillentry[*][<post_padding>]{<skill_cathegory>}{<0-5>}{<skill_name>}{<years_of_experience>}{<comment>}%
\DeclareDocumentCommand\cvskillentry{s +O{} +m +m +m +m +m}{}%
%     add cvskill matrix row.
% 
%     Input:
%         asterix (optional): include horizontal (dashed) line above the entered line. This behaviour depends on the body style. 
%                             For the fancy style, the asterix has no meaning.
%         input_1 (optional): padding length appended to the legend, default: <0.25em>
%         input_2: string, naming skill cathegory, default: <>
%         input_3: integer between 0 and 5, describing level of skill. \cvskill{input_2} is called internally, default: <>
%         input_4: string, naming the skill, default: <>
%         input_5: positive real number, stating the number of years of experience with this skill , default: <>
%         input_6: string, explaining details w.r.t. that particual skill default: <>
% 
%     Example usages:
%         \cvskillentry*{Language:}{3}{Python}{2}{I have done a million projects with Python}
%         \cvskillentry{}{2}{Lilypond}{14}{So much sheet music! Man I'm the best!}
%         \cvskillentry{}{3}{\LaTeX}{14}{Clearly I rock at \LaTeX}
%         \cvskillentry*[1.5em]{OS:}{3}{Linux}{2}{I only use Archlinux}
% 
%     Note:   
%         - The width of the columns can be adjusted by the \setcvskillcolumns command, see \setcvskillcolumns.

% Definition of \cvskillentry
%\RenewDocumentCommand\cvskillentry{s +O{.25em} +m +m +m +m +m}{%
%    %test for the star * in the command
%    \IfBooleanTF{#1}{% If a star is seen a dotted line is drawn above the entry
%        \begingroup
%            \renewcommand{\arraystretch}{1.1}
%            \arrayrulecolor{awesome}
%            % [*][<post_padding>]{<skill_category>}{<0-5>}{<skill_name>}{<years_of_experience>}{<comment>}%
%            \begin{tabular}{| p{\skillmatrix@hintscolumnwidth}@{\hspace{\separatorcolumnwidth}} | p{\cvskill@width}@{\hspace{\skillmatrix@padding}} | p{\cvskill@descriptorwidth}@{\hspace{\skillmatrix@padding}} | p{\cvskill@experiencewidth}@{\hspace{\skillmatrix@padding}} | p{\skillmatrix@commentwidth}@{} | } %
%                \cdashline{2-5}[.6pt/1pt] %
%                \raggedleft \entrypositionstyle{#3} & \centering \cvskill{#4} & \centering {\skilldetailstyle{#5}} & \centering \skilldetailstyle{#6} & {\skilldetailstyle{ #7}}%
%                %\raggedleft \hintstyle{#3} & \centering \cvskill{#4} & \centering {\skilldetailstyle{#5}} & \centering \skilldetailstyle{#6} & {\skilldetailstyle{ #7}}%
%            \end{tabular}%
%        \endgroup
%        \par\addvspace{#2}
%    }{
%% If no star is seen no line is drawn
%\begin{tabular}{|p{\skillmatrix@hintscolumnwidth}@{\hspace{\separatorcolumnwidth}} | %
%p{\cvskill@width}@{\hspace{\skillmatrix@padding}} | %
%p{\cvskill@descriptorwidth}@{\hspace{\skillmatrix@padding}} | %
%p{\cvskill@experiencewidth}@{\hspace{\skillmatrix@padding}} | %
%p{\skillmatrix@commentwidth}@{} |}%
%%\raggedleft \skilltypestyle{#3} & \centering \cvskill{#4} & \centering {\skilldetailstyle{#5}} & \centering \skilldetailstyle{#6} & {\skilldetailstyle{ #7}}%
%\raggedleft \entrypositionstyle{#3} & \centering \cvskill{#4} & \centering {\skilldetailstyle{#5}} & \centering \skilldetailstyle{#6} & {\skilldetailstyle{ #7}}%
%
%\end{tabular}%
%\par\addvspace{#2}
%}
%}


    \DeclareDocumentCommand\@starIndependentMatrixEntry{}{}%
    \RenewDocumentCommand\cvskillentry{s O{.25em} +m +m +m +m +m}{%
        \arrayrulecolor{awesome}%
        \setlength\arrayrulewidth{\separatorrulewidth}%
        \vspace*{-\separatorrulewidth}% HACK; I don't understand where the space is coming from, nor what it's exact value is :(
        %test for the star * in the command. 
        \RenewDocumentCommand{\@starIndependentMatrixEntry}{}{%
        %\fbox{
            \begingroup%
                % \renewcommand{\arraystretch}{1.25}%
%                \addvbuffer[1pt 2pt]
                \begin{tabular}[c!]{ | @{} p{\hintscolumnwidth}@{\hspace{\skillmatrix@padding}}%
                                %@{\hspace{\separatorcolumnwidth}}@{\hspace{\separatorcolumnwidth}} 
                                         | p{\skillmatrix@hintscolumnwidth} @{\hspace{\skillmatrix@padding}}%
                                         | p{\cvskill@width}@{\hspace{\skillmatrix@padding}} %
                                         | p{\cvskill@descriptorwidth}@{\hspace{\skillmatrix@padding}}%
                                         | p{\cvskill@experiencewidth}@{\hspace{\skillmatrix@padding}} | p{\skillmatrix@commentwidth}@{}| }%
                     %\cline{3-6}%
    %\hline
                    & \entrypositionstyle{#3}\hspace*{3\skillmatrix@padding} & \centering \cvskill{#4} & \centering \skilldetailstyle {#5} &  \centering \skilldetailstyle{#6} & \skilldetailstyle{#7} \\[#2]%                  
                    % \skilldetailstyle{#7} \\[#2]%
                \end{tabular}%
            \endgroup%
            %}
        }%
        \IfBooleanTF{#1}{% the star does not do anything here
            \@starIndependentMatrixEntry%
        }{% 
            \@starIndependentMatrixEntry%
        }%
        \\[-0.5\baselineskip] %\newline\@aftersectionfalse\ignorespaces%
    }%



% \setcvskillcolumns[][0.3][2em]

\NewDocumentCommand\skillLegend@FontSize{}{\scriptsize}

\DeclareDocumentCommand\cvskilllegend{s +O{} +O{} +O{} +O{} +O{} +O{} +m}{}%

\NewDocumentCommand\skillLegend@defaultLevelOne{}{basic knowledge}
\NewDocumentCommand\skillLegend@defaultLevelTwo{}{intermediate knowledge with some project experience}
\NewDocumentCommand\skillLegend@defaultLevelThree{}{extensive project experience}
\NewDocumentCommand\skillLegend@defaultLevelFour{}{deepened expert knowledge}
\NewDocumentCommand\skillLegend@defaultLevelFive{}{expert\,/\,specialist}

\RenewDocumentCommand\cvskilllegend{s +O{.25em} +O{\skilllegend@defaultLevelOne} +O{\skillLegend@defaultLevelTwo} +O{\skillLegend@defaultLevelThree} +O{\skillLegend@defaultLevelFour} +O{\skillLegend@defaultLevelFive} +m}{%
    \IfBooleanTF#1{% if a star is given, add dashed line
        \begingroup%
            \arrayrulecolor{red}%
            % calculate descriptor widths
% left width            
            \setlength{\cvskilllegend@leftdescriptorwidth}{6cm}
            %\skilllegend@leftdesriptorfactor\skilllegend@bodylength-\cvskill@width-\skillmatrix@padding-3\skilllegend@padding}%
% right width            
            \setlength{\cvskilllegend@rightdescriptorwidth}{5cm}%\skilllegend@bodylength-\skilllegend@leftdesriptorfactor\skilllegend@bodylength-\cvskill@width-\skillmatrix@padding-3\skilllegend@padding}%
            % table
            \begin{tabular}{ %| @{}p{\skilllegend@hintscolumnwidth}@{\hspace{\separatorcolumnwidth}}%
                             %  | p{\cvskill@width}@{\hspace{\skilllegend@padding}};{.6pt/1pt}%
                             %  | p{2\skilllegend@padding} %
                             %  | p{\cvskilllegend@leftdescriptorwidth}@{}@{\hspace{2\skillmatrix@padding}}%
                             %  | p{\cvskill@width}@{\hspace{\skilllegend@padding}};{.6pt/1pt}%
                             %  | p{2\skilllegend@padding}%
                             %  | p{\cvskilllegend@rightdescriptorwidth}@{}
%                               
%                              
%| p{\skillmatrix@hintscolumnwidth} %@{} p{\hintscolumnwidth}@{\hspace{\skillmatrix@padding}}% 0
%
p{\skillmatrix@hintscolumnwidth} @{\hspace{\skillmatrix@padding}} %
| p{\skillmatrix@hintscolumnwidth} @{\hspace{\skillmatrix@padding}}% 1
| p{5cm} % 2
| p{\skillmatrix@hintscolumnwidth} @{\hspace{\skillmatrix@padding}}% 3  
| p{5cm} | %4                  
                               }%
% first line
                \cdashline{2-3}[6pt/1.5pt]
                &  \cvskill{1} & \skilldetailstyle{basic knowledge}                  & \cvskill{3} & \skilldetailstyle{ #5} \\
                      &  \cvskill{2} & \multirow{2}{5cm}{\skilldetailstyle{ #4}} & \cvskill{4} & \skilldetailstyle{ #6} \\
                      &              &                                         & \cvskill{5} &  \skilldetailstyle{ #7}% 
                
%                \raggedleft\hintstyle{{\bfseries#8}} & \cvskill{1} & & {\skilllegend@FontSize #3} & \cvskill{3} & & {\skilllegend@FontSize #5} \\ %
%%
%% second
%                 & \cvskill{2} & & \multirow{2}{\cvskilllegend@leftdescriptorwidth}{{\skilllegend@FontSize #4}} & \cvskill{4} & & {\skilllegend@FontSize #6} \\%
%                    %         
%% third
%                 & & & & \cvskill{5} & & {\skilllegend@FontSize #7}%


% first line
%                \cdashline{2-7}[6pt/1.5pt]
%                \raggedleft\hintstyle{{\bfseries #8}} & 1 & 2 & 3 & 4 & 5 & \skilldetailstyle{ #6} \\ %
%
% second
%                 0 & 1 & 2 & 3 &  \fbox{\cvskill{4}} & \skilldetailstyle{ #5} & 6 \\%
%                    %         
% third
%                 experience0 & experience1 & experience2 & experience3 & experience4 & experience5 & \skilldetailstyle{ #7} %

            \end{tabular}%
        \endgroup
        % second part, without star
        \par\addvspace{#2}}{%
        % if no star is given, do not add dashed line. We need less padding in this case
        \begingroup%
            % calculate descriptor columns width. note the adjusted padding
            \setlength{\cvskilllegend@leftdescriptorwidth}{\skilllegend@leftdesriptorfactor\skilllegend@bodylength-\cvskill@width-\skillmatrix@padding-1\skilllegend@padding}%
            \setlength{\cvskilllegend@rightdescriptorwidth}{\skilllegend@bodylength-\skilllegend@leftdesriptorfactor\skilllegend@bodylength-\cvskill@width-\skillmatrix@padding-1\skilllegend@padding}%
            \begin{tabular}{@{}p{\skilllegend@hintscolumnwidth}%
                            @{\hspace{\separatorcolumnwidth}}%
                            p{\cvskill@width}@{\hspace{\skilllegend@padding}}%
                            p{\cvskilllegend@leftdescriptorwidth}@{\hspace{2\skillmatrix@padding}}%
                            p{\cvskill@width}@{\hspace{\skilllegend@padding}}%
                            p{\cvskilllegend@rightdescriptorwidth}@{}}%
                \raggedleft\hintstyle{{\bfseries#8}}  & \cvskill{1} \, & \, {\skilllegend@FontSize #3} & \cvskill{3} \, & \,{\skilllegend@FontSize #5 } \\%
                            %
                        & \cvskill{2} \, & \, \multirow{2}{\cvskilllegend@leftdescriptorwidth}{{\skilllegend@FontSize #4}} & \cvskill{4}\, & \,{\skilllegend@FontSize #6 } \\%
                    %         
                        &  &   & \cvskill{5} \, & \, {\skilllegend@FontSize #7}%
            \end{tabular}%
        \endgroup%
        \par\addvspace{#2}%
    }%
}%

\makeatother
%%%%%%%%%%%%%%%%

% 1 : skill type
% 2 : columns 2,3,4
% 3 : size of year 
\setcvskillcolumns[\widthof{ \entrypositionstyle{Presentations111}} ][0.425][2em]