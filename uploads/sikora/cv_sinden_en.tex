% -- Encoding UTF-8 without BOM
% -- XeLaTeX => PDF (BIBER)

\documentclass[english, print]{cv-style-wide-en}     
     
% Add 'print' as an option into the square bracket to remove colours from this template for printing. 
% Add 'espanol' as an option into the square bracket to change the date format of the Last Updated Text

%\sethyphenation[variant=british]{english}{Uniform} % Add words between the {} to avoid them to be cut 

% implemented, lead, revolutionised, leveraged, conduct, explore, manage,% detail (or results) orientated / accomplihsed / skilled / experienced / adapt, conduct , explore


%LANGUAGES               | SOFTWARE
%Native :                | Libraries and Packages
%English, German         | PyTorch • PyTorchLightning • Numpy
%Advanced :              | k-Wave • CUDA • OpenMPI • OpenMP
%Python, MATLAB, Bash    | Tools
%Intermediate :          | Linux • Docker • profiling • vim • tmux •Kubernetes
%Julia, C++              | Polyaxon • PyCharm • CLion • CMake




% https://www.overleaf.com/latex/templates/faangpath-simple-template/npsfpdqnxmbc

\usepackage{changepage} 

%\begin{hyphenrules}{english}
%\hyphenation{Uniform}
%\end{hyphenrules}


\usepackage[unicode,
            colorlinks=false,
            bookmarks=true,
            bookmarksopen=true,
            bookmarksdepth=2
           ]{hyperref}
\hypersetup{
    pdftitle=CV David Sinden,
    pdfauthor=David Sinden,
    pdfsubject={CV, Resume},
    colorlinks=false,
    linkbordercolor=blue,
    urlbordercolor=red
}

\begin{document}

\settowidth{\bwidth}{BSc}
\settowidth{\mwidth}{MSc}
\settowidth{\pwidth}{PhD}

\settowidth{\thesiswidth}{Thesis:}
\settowidth{\advisorwidth}{Advisor:}
\setlength{\uclwidth}{\advisorwidth - \thesiswidth}

\setlength{\diffbwidth}{\mwidth - \bwidth}
\setlength{\diffpwidth}{\mwidth - \pwidth}

\header{David }{Sinden}{}


\begin{tikzpicture}[remember picture,overlay]
  \node[
    anchor=north west,
  ] at ([yshift=-0.25cm, xshift=0.8cm]current page.north west)
  {\includegraphics[width=2.5cm]{../img/profile.png}};  
  \node[
    anchor=north west,
    xshift=3.5cm,                          % 2.5cm(photo)+0.5cm margin
    yshift=-1.48cm,
    text width=\dimexpr\paperwidth-3cm-1.5cm\relax, % leave 1cm right margin
    align=left,                        % or center or flush left/right as you like
    text=MidnightBlue,
    font=\thinfont
  ] at (current page.north west) {%
    home: \href{https://djps.github.io}{djps.github.io} \quad
    \href{https://github.com/djps}{github: djps} \quad
    \href{https://www.linkedin.com/in/sindendavid/}{linkedin: sindendavid} \quad
    \href{https://www.xing.com/profile/David_Sinden}{xing: david\_sinden} \quad
    % \href{https://x.com/david_sinden}{twitter: @david\_sinden} \quad
    \href{mailto:david.sinden@gmail.com}{david.sinden@gmail.com} \\
    Bremen, \quad British Citizen with permanent residency in Germany, \quad 27.12.1981
  };  
\end{tikzpicture}       
        
\lastupdated


%----------------------------------------------------------------------------------------
%	SKILLS SECTION
%----------------------------------------------------------------------------------------

%\vfill

%\section{Skills}

\vspace{0.5cm}  

% Experienced scientist with a demonstrated history of developing product-level code and algorithms in close collaboration with business partners and end customers. Over 10 years of experience in the domain of external beam radiation therapy, developing automatic segmentation algorithms and advanced algorithms for plan optimization. Recent projects on anatomical mapping for cardiac electro-physiology as well as cone-beam reconstruction for c-arm systems. Over 20 years of experience using and developing statistical machine learning methods. I master C++, python, *nix, work agile, believe in test-driven development, CI/CD, and quickly learn tools that help improve efficiency. Basic knowledge in containerization, cuda, PyTorch. My interests are to work on developing machine learning methods in combination with (advanced) physical and mathematical modeling - together with friendly and collaborative colleagues.

% I have been researching methods in the field of medical imaging systems and their clinical application for more than 20 years. I focus on classical image processing, machine learning and data analysis. My strengths lie in my ability to quickly familiarize myself with related fields, to structure complex situations and to recognize connections between different disciplines. I find it easy to shape the dialog between scientists, IT developers and customers. This results in new perspectives and inn

%\item[\checkmark] Implemented XYZ method - Accomplished [X] as measured by [Y], by doing [Z] 
%    \item[\checkmark] Developed XYZ using ABC that led to X\% improvement 
%    \item[\checkmark] Solved ABC with AAA method leading to CCC.
%    \item[\checkmark] Implemented XYZ library for ABC.
%    \item[\checkmark] Utilized XYZ that increased AA by x\%
\setulcolor{MidnightBlue}%
\setul{0.5ex}{0.4pt}% instead of strong track record/ proven expertise in
Innovative applied mathematician with a strong track record in {scientific computation}, with over 15 years' experience working in applied research with a focus on medical interventions, internationally recognised expert on ultrasound simulation for therapy and diagnosis. Looking for new challenges in modelling to support medical device development, open to relocation.
\begin{NoHyphItemize}
\item Formulates realistic yet tractable mathematical models, and deployable, tested and documented simulations.  Excellent {coding} and {software development} skills developed through clinical/commercial deployment, satisfying ISO 13485/IEC 62304 standards.  
%
\item Communication skills refined through working in multi-disciplinary, international teams at the interface of academic research and industry; lecturing and teaching at universities; delivering invited presentations at international conferences. Experienced in leading a managing complex interdisciplinary projects, as well as working in a team.
 
\item 17 \href{https://scholar.google.co.uk/citations?user=lauJqakAAAAJ&hl=en}{peer-reviewed journal papers} (over 250 citations), a book chapter, acquisition of grants (over 500,000\euro), 12 invited presentations, industrial supervisor to three PhD students, an MSc student, and maintainer of widely used open source code, \href{https://k-wave-python.readthedocs.io/en/latest/index.html#citation}{{k-wave-python}}
\end{NoHyphItemize}
\vspace{1\baselineskip}
  
%----------------------------------------------------------------------------------------
%	WORK EXPERIENCE SECTION
%----------------------------------------------------------------------------------------
% \vfill

\section{Professional Experience}
\pdfbookmark[1]{Experience}{sec:experience}
%\vspace{-0.2cm}
\begin{entrylist}
%------------------------------------------------

% 

\entry
  {Nov 2019--}
  {Fraunhofer-Institut für Digitale Medizin MEVIS}
  {Bremen, Germany}
  {\jobtitle{Senior Research Scientist~\textemdash~\href{https://www.mevis.fraunhofer.de/en/research-and-technologies/modeling-and-simulation.html}{Modelling \& Simulation Group}\\%
  \text{\normalfont{Skills: python, VTK, ultrasound modelling, treatment planning, software development, elastography}}, \newline{}\text{\normalfont{uncertainty analysis} }}
  \begin{itemize}[topsep=0pt]
\item Developed large-scale simulations for microwave and ultrasound ablative therapies  by architecting and parallelizing high-performance numerical methods, enabling clinically relevant treatment planning at scale.
%
\item Engineered fast ultrasound beamforming algorithms and a transcranial acoustic/elastic propagation simulator  by leveraging GPU-accelerated signal-processing techniques, delivering real-time imaging performance and improved diagnostic accuracy.
  \end{itemize}}
%------------------------------------------------

\entry
  {Jun 2014--Nov 2019}
  {National Physical Laboratory}
  {Teddington, United Kingdom}
  {\jobtitle{Senior Research Scientist~\textemdash~\href{https://www.npl.co.uk/research/ultrasound/therapeutic}{Ultrasound \& Underwater Acoustics Group}\\%
  \text{\normalfont{\textcolor{red}{Skills: python, matlab, Finite Element Analysis (COMSOL, FeniCS), ultrasound modelling, signal processing} }}}
  \begin{itemize}
\item Established measurement-based simulation for nonlinear propagation through complex media by integrating empirical measurements into high-fidelity nonlinear computational models, enabling accurate predictions and incorporation into the IEC standard 63587.
  \end{itemize}}
%------------------------------------------------



\entry
  {Jun 2011-- Jun 2014}
  {Institute of Cancer Research/The Royal Marsden Hospital}
  {Sutton, United Kingdom}
  {\jobtitle{Post Doctoral Research Associate~\textemdash~\href{https://www.icr.ac.uk/research-and-discoveries/icr-divisions/radiotherapy-and-imaging/therapeutic-ultrasound}{Therapeutic Ultrasound Group}/Joint  Department of Physics\\%
  \text{\normalfont{\textcolor{red}{Skills: python, VTK, ultrasound modelling, treatment planning, software development} }}}
    \begin{itemize}
    \item Developed an ultrasound-guided high-intensity focused ultrasound treatment planning system with a multi-element phased-array by integrating real-time ultrasonic imaging and adaptive beamforming control, enabling precise focus steering, dynamic treatment adjustments, and enhanced therapy safety.
    \end{itemize}}
%------------------------------------------------

% 

\entry
  {Jun 2008-- Jun 2011}
  {University College London}
  {London, United Kingdom}
  {\jobtitle{Post Doctoral Research Associate~\textemdash~\href{https://www.ucl.ac.uk/engineering/ultrasonics-group}{Ultrasound Group}/Department of Mechanical Engineering\\%
  \text{\normalfont{Skills: mathematical modelling, Fortran, differential equations} }}
  	\begin{itemize}
      \item Investigated the influence of cavitation on therapeutic ultrasound by employing numerical simulations and analytical modeling, enabling accurate prediction of cavitation thresholds and optimization of treatment efficacy and safety.
    \end{itemize}}
%------------------------------------------------

\end{entrylist}

%----------------------------------------------------------------------------------------
%	EDUCATION SECTION
%----------------------------------------------------------------------------------------
% \vfill

%\vspace{-0.2cm}
\section{Education}%\color{green}\vhrulefill{0.9pt}
\pdfbookmark[1]{Education}{sec:education}
%\vspace{-0.2cm}
%------------------------------------------------
\begin{entrylist}
%\vspace*{-0.5cm}
\threeentry{2004--2008}%
{PhD \hspace{\diffpwidth/2}-\hspace{\diffpwidth/2} Dynamical Systems}%
{%
%\setulcolor{MidnightBlue}%
%\setul{0.5ex}{0.4pt}%
\href{https://ucl.ac.uk}{University College London, United Kingdom}%
}%
{}%
%\setulcolor{MidnightBlue}%
%\setul{0.5ex}{0.4pt}%
% Thesis: \hspace{\uclwidth}\href{https://www.google.com/url?sa=t&source=web&rct=j&opi=89978449&url=https://discovery.ucl.ac.uk/15832/1/15832.pdf}{\enquote{Integrability, Localisation and Bifurcation of an Elastic Conducting Rod in a Uniform Magnetic Field}} %,\newline{}Advisor: \href{https://profiles.ucl.ac.uk/12184-gert-van-der-heijden}{Prof. Gert van der Heijden}%

%}

%------------------------------------------------
%\vspace*{-0.5cm}
\threeentry{2003--2004}%
{MSc - Modern Applications of Mathematics}%
{%
%\setulcolor{MidnightBlue}%
%\setul{0.5ex}{0.4pt}%
\href{https://bath.ac.uk}{University of Bath, United Kingdom}%
}%
{}%

%------------------------------------------------
%\vspace*{-0.5cm}
\threeentry{2000--2003}%
{BSc \hspace{\diffbwidth/2}-\hspace{\diffbwidth/2} Mathematics with Applied Math./Math. Physics}%
{%
%\setulcolor{MidnightBlue}%
%\setul{0.5ex}{0.4pt}%
\href{https://imperial.ac.uk}{Imperial College London, United Kingdom}%
}%
{}%

%------------------------------------------------
\end{entrylist}



%----------------------------------------------------------------------------------------
%	AWARDS SECTION
%----------------------------------------------------------------------------------------
%\vfill

\newpage
\section{Auszeichnungen \& Wertschätzungsindikatoren}
% \addcontentsline{toc}{Awards}{sec:awards}
\pdfbookmark[1]{Auszeichnungen}{sec:awards}
%\vspace{-0.2cm}
\begin{entrylist}
%------------------------------------------------
\entry
{2020}
{IEEE IUS Challenge on Ultrasound Beamforming with Deep Learning (CUBDL)}
{~}
%{Joint~\newhl{first place}~in IEEE IUS Challenge for \enquote{Improving image quality of single plane wave ultrasound via deep learning based channel compounding} (2020) }
% {Joint~first place~in IEEE IUS \href{https://cubdl.jhu.edu/}{CUBDL} Challenge for \enquote{Improving image quality of single plane wave ultrasound via deep learning based channel compounding} (2020) }
{Joint first place in international machine learning challenge applied to ultrasound image reconstruction (2020)}

\entry
{2015-}
{International Expert}
{~}
{Member \href{https://iec.ch/dyn/www/f?p=103:7:::::FSP_ORG_ID:1281}{IEC}/\href{https://standardsdevelopment.bsigroup.com/committees/50001534}{BSI} Technical Committee 87 (Ultrasonics), in an individual capacity, associate \href{https://ima.org.uk/}{IMA}, full member \href{https://www.iop.org/}{IOP}}
%
%
%\entry
%{2020}
%{Award name}
%{~}
%{Award description. Award description. Award description. Award description. Award description. Award description. Award description. }
%
\entry
{Various}
{Enhanced Scholarships}
{~}
{Enhanced funding for MSc (2004), PhD (2008) and post-doctoral work (2014) from UK funding agency EPSRC}
%------------------------------------------------
\end{entrylist}

%----------------------------------------------------------------------------------------
%	INTERESTS SECTION
%----------------------------------------------------------------------------------------

\section{Skils}\pdfbookmark[1]{Skills}{sec:skills}
  %\vspace{-0.2cm}
\begin{tabular}{p{0.6\textwidth} p{0.40\textwidth} }
\textbf{Programming:} python, C++, Matlab, OpenCL & \textbf{Libraries:} ITK, VTK, boost, eigen \\
%
\textbf{DevOps:} git, svn, github, gitlab, google test, pytest, make, cmake, visual studio & \textbf{Computation:} FEniCS, Comsol \\
%
%\textbf{testing}:  & \textbf{documentation}: sphinx, doxygen \\
% \textbf{build systems}: make, cmake, visual studio  & \textbf{presentation}: \LaTeX{}, Bib\TeX{}, html, css: tailwind, bootstrap \\
%
 \textbf{Languages:} Englisch (Native), Deutsch (B2.1) with permanent residency & {}
\end{tabular}





%----------------------------------------------------------------------------------------
%	INTERESTS SECTION
%----------------------------------------------------------------------------------------
%\vfill
%
%\section{Referees}
%  \vspace{-0.2cm}
%Available on request.
%----------------------------------------------------------------------------------------
\end{document}