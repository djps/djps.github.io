%!TEX TS-program = xelatex
%!TEX encoding = UTF-8 Unicode
% Awesome CV LaTeX Template for CV/Resume
%
% This template has been downloaded from:
% https://github.com/posquit0/Awesome-CV
%
% Author:
% Claud D. Park <posquit0.bj@gmail.com>
% http://www.posquit0.com
%
% Also see
% https://github.com/liantze/AltaCV/blob/main/sample.png
%
% Adapted to be an Rmarkdown template by Mitchell O'Hara-Wild
% 23 November 2018
%
% Template license:
% CC BY-SA 4.0 (https://creativecommons.org/licenses/by-sa/4.0/)
%
%-------------------------------------------------------------------------------
% CONFIGURATIONS
%-------------------------------------------------------------------------------
% A4 paper size by default, use 'letterpaper' for US letter.
\documentclass[11pt, a4paper]{awesome-cv}

% Configure page margins with geometry
\geometry{left=1.4cm, top=0.8cm, right=1.4cm, bottom=1.8cm, footskip=.5cm}

% Specify the location of the included fonts
%\fontdir[/usr/share/texlive/texmf-dist/fonts/]

\usepackage{fontspec}

% Set Raleway as the main font
\setmainfont{Raleway}

% Color for highlights
% Awesome Colors: awesome-emerald, awesome-skyblue, awesome-red, awesome-pink, awesome-orange
%                 awesome-nephritis, awesome-concrete, awesome-darknight

% this is teal
\definecolor{awesome}{HTML}{008080}

% tailwind-cyan-500
\definecolor{awesome}{HTML}{06b6d4}
% tailwind-cyan-600
\definecolor{awesome}{HTML}{0891b2} 
\definecolor{awesome}{RGB}{8, 145, 178}

\definecolor{awesome-faded}{RGB}{85, 222, 255}



% Colors for text
% Uncomment if you would like to specify your own color
% \definecolor{darktext}{HTML}{414141}
% \definecolor{text}{HTML}{333333}
% \definecolor{graytext}{HTML}{5D5D5D}
% \definecolor{lighttext}{HTML}{999999}

% Set false if you don't want to highlight section with awesome color
\setbool{acvSectionColorHighlight}{false}

% This overwrites the boolean above
\makeatletter
\def\@sectioncolor{\color{awesome}}
\makeatother

% If you would like to change the social information separator from a pipe (|) to something else
\renewcommand{\acvHeaderSocialSep}{\quad\textbar\quad}

\def\endfirstpage{\newpage}

%-------------------------------------------------------------------------------
%	PERSONAL INFORMATION
%	Comment any of the lines below if they are not required
%-------------------------------------------------------------------------------
% Available options: circle|rectangle, edge/noedge, left/right

\photo{profile.png}

\name{David Sinden}{}

\position{Applied Mathematician --- Research Software Engineer}

\email{\href{mailto:david.sinden@gmail.com}{\nolinkurl{david.sinden@gmail.com}}}
\homepage{djps.github.io}
\github{djps}
\linkedin{sindendavid}
\twitter{david\_sinden}
% \orcid{0000-0002-8514-8279} % need to see if this can be re-ordered 456 of cls.

% \gitlab{gitlab-id}
% \stackoverflow{SO-id}{SO-name}
% \skype{skype-id}
% \reddit{reddit-id}

\usepackage{booktabs}
\usepackage{csquotes}

\newlength{\xbigstrutjot}
\setlength{\xbigstrutjot}{1em}

\makeatletter
\@ifundefined{xbigstrutjot}{\newdimen\xbigstrutjot}{} 
\xbigstrutjot=10pt

\newcommand\xbigstrut[1][x]{%
  \leavevmode\unskip\@tempdima=\ht\@arstrutbox \@tempdimb=\dp\@arstrutbox
  \ifx #1b\relax \else \advance\@tempdima by \xbigstrutjot\fi
  \ifx #1t\relax \else \advance\@tempdimb by \xbigstrutjot\fi
  \hbox{\textcolor{blue}{\vrule} \@height\@tempdima \@depth\@tempdimb \@width\z@}\ignorespaces}
\makeatother

 \newcommand{\customstrut}{\textcolor{blue}{\rule{0pt}{0.5em}}}

\usepackage[calc]{datetime2}

\DTMsavenow{mytoday}

\providecommand{\tightlist}{\setlength{\itemsep}{0pt}\setlength{\parskip}{0pt}}

%------------------------------------------------------------------------------

% Pandoc CSL macros
\newlength{\cslhangindent}
\setlength{\cslhangindent}{1.5em}
\newlength{\csllabelwidth}
\setlength{\csllabelwidth}{3em}
\newenvironment{CSLReferences}[3] % #1 hanging-ident, #2 entry spacing
 {% don't indent paragraphs
  \setlength{\parindent}{0pt}
  % turn on hanging indent if param 1 is 1
  \ifodd #1 \everypar{\setlength{\hangindent}{\cslhangindent}}\ignorespaces\fi
  % set entry spacing
  \ifnum #2 > 0
  \setlength{\parskip}{#2\baselineskip}
  \fi
 }%
 {}
 
\usepackage{calc}
\newcommand{\CSLBlock}[1]{#1\hfill\break}
\newcommand{\CSLLeftMargin}[1]{\parbox[t]{\csllabelwidth}{#1}}
\newcommand{\CSLRightInline}[1]{\parbox[t]{\linewidth - \csllabelwidth}{#1}}
\newcommand{\CSLIndent}[1]{\hspace{\cslhangindent}#1}

\usepackage{colortbl}

\usepackage{printlen}

%%%%%%%%%%%%%%%%
\RequirePackage{tikz}
\usepackage{adjustbox}
\usepackage{mdframed}

\usepackage{verbatimbox}

\RequirePackage{multirow}
\RequirePackage{arydshln} % incompatible with fancy style

\makeatletter

\newcommand*{\@initializelength}[1]{%
  \ifdefined#1
  \else%
    \newlength{#1}\fi%
  \setlength{#1}{0pt}}

% creates an if switch if not yet defined
\newcommand*{\@initializeif}[1]{%
%  \ifdefined#1% not working due to the nested \if
%  \else%
    \newif#1%\fi
  }

\@initializeif{\if@aftersection}\@aftersectionfalse%

% definitions form elsewhere
\newcommand*{\hintfont}{}
\newcommand*{\hintstyle}[1]{{\hintfont#1}}

\renewcommand*{\hintstyle}[1]{{\hintfont\textcolor{awesome}{#1}}}

\@initializelength{\separatorcolumnwidth}         
\setlength{\separatorcolumnwidth}{0.025\textwidth}

\@initializelength{\hintscolumnwidth}             
\setlength{\hintscolumnwidth}{0.0\textwidth}

\@initializelength{\maincolumnwidth}
\setlength{\maincolumnwidth}{\textwidth-\leftskip-\rightskip}%

\@initializelength{\cvSkill@RectangleSize}
\setlength{\cvSkill@RectangleSize}{1.2ex}
\newcount\my@repeat@count

\DeclareDocumentCommand{\cvskill}{m}{%
%   Illustrate skill level with little colored boxes.
%   By default five skill levels are predefined.   
% 
%   Input: 
%       integer between 0 and 5
%   Example usage: 
%       \cvskill{3}
\smash{\raisebox{-0.65\baselineskip}{
%\fbox{
    \begingroup
        \my@repeat@count=\z@
        \@whilenum\my@repeat@count<#1\do{\tikz\filldraw[awesome] (0, 0) rectangle (\cvSkill@RectangleSize, \cvSkill@RectangleSize);\advance%
        \my@repeat@count\@ne\,}%
        \my@repeat@count=\numexpr5-\z@\relax
        \@whilenum\my@repeat@count>#1\do{\tikz\filldraw[awesome-faded] (0, 0) rectangle (\cvSkill@RectangleSize, \cvSkill@RectangleSize);\advance%
        \my@repeat@count\m@ne\,}%
    \endgroup
    %}
    }}
}% end \cvskill

\@initializelength{\cvskill@width}
\@initializelength{\cvskill@descriptorwidth}
\@initializelength{\cvskill@experiencewidth}

\@initializelength{\skillmatrix@columnwidth}
\@initializelength{\skillmatrix@commentwidth}
\@initializelength{\skillmatrix@padding}
\@initializelength{\skillmatrix@hintscolumnwidth}
\@initializelength{\skillmatrix@bodylength}

\@initializelength{\cvskilllegend@leftdescriptorwidth}
\@initializelength{\cvskilllegend@rightdescriptorwidth}

\@initializelength{\skilllegend@hintscolumnwidth}
\@initializelength{\skilllegend@padding}
\@initializelength{\skilllegend@bodylength}

\@initializelength{\separatorrulewidth}

\DeclareDocumentCommand{\skilllegend@leftdesriptorfactor}{}{}%

%% DEFINITION \recompute@cvskillmatrix@lengths
% declare the command \recompute@cvskillmatrix@lengths empty

\DeclareDocumentCommand{\recompute@cvskillmatrix@lengths}{}{}%

\RenewDocumentCommand{\recompute@cvskillmatrix@lengths}{}{%

\setlength{\skillmatrix@padding}{1ex}%

\setlength{\skillmatrix@hintscolumnwidth}{\hintscolumnwidth}%

\setlength{\cvskill@width}{\widthof{\cvskill{5}}}%

\setlength{\cvskill@experiencewidth}{\widthof{``Year''}}%

\setlength{\skillmatrix@bodylength}{\maincolumnwidth}%

\setlength{\skillmatrix@columnwidth}{0.45\skillmatrix@bodylength}%

\setlength{\cvskill@descriptorwidth}{\skillmatrix@columnwidth-\cvskill@width-\cvskill@experiencewidth}%

\setlength{\skillmatrix@commentwidth}{\skillmatrix@bodylength-\skillmatrix@columnwidth-\skillmatrix@padding}%

% lengths and definitions needed for the legends
% note that \skillmatrix@padding also affects the appearance of legends
\setlength{\skilllegend@padding}{0.25ex}%

\setlength{\skilllegend@hintscolumnwidth}{\hintscolumnwidth}%

\setlength{\skilllegend@bodylength}{\skillmatrix@bodylength}%

\RenewDocumentCommand{\skilllegend@leftdesriptorfactor}{}{0.5}%

%\setlength{\cvskill@descriptorwidth}{5cm}%
%\setlength{\skillmatrix@commentwidth}{10cm}%
%\setlength{\skilllegendbodylength}{\skillmatrix@bodylength}%
}

% \setcvskillcolumns[<width>][<factor>][<exp_width>]
\DeclareDocumentCommand{\setcvskillcolumns}{+O{\skillmatrix@hintscolumnwidth} +O{\skillmatrix@columnwidth} +O{\cvskill@experiencewidth}}{%
%   adjust column width of skill table
% 
%   The \cvskillentry command comes with default FIXED width definitions of the kill matrix for the columns 
%   based on the author's own skill matrix. This is necessary to ensure that the entries are nicely aligned 
%   and actually yield a decent looking table. The defaults depend on the style used and are chosen reasonably.
%   However, depending on the user input and the style that is used some of the columns might need adjustments. 
%   The \setcvskillcolumns command provides means to influence the width of the first, the third and the fourth 
%   skill matrix column. The second column containg the output of \cvskill remains fixed width. The last column,
%   the comment column gets recalculated according to the setting of the other columns.
% 
%   Input
%       Input_1 (optional):     width smaller than \textwidth,  default <\skillmatrix@hintscolumnwidth>
%       Input_2 (optional):     float between 0 and 1 adjusting how much percent of the table width without 
%                               the first column is used columns two, three and four. Through this parameter
%                               the width of the 3rd column (skill name) can be adjusted,   default <\skillmatrix@columnwidth>
%       Input_3 (optional):     width smaller than \textwidth, setting the width of the 4th column 
%                               (Years of experience), default <\cvskill@experiencewidth>
% 
%   Example usage
%       \setcvskillcolumns[5em][][]%    adjust first column. Same as \setcvskillcolumns[5em]
%       \setcvskillcolumns[][0.45][]%   adjust third (skill) column. Same as \setcvskillcolumns[][0.45]
%       \setcvskillcolumns[][][\widthof{``Year''}]%     adjust fourth (years) column.
%       \setcvskillcolumns[\widthof{``Language''}][0.48][]%     adjust 1st and 3rd columns. Same as \setcvskillcolumns[\widthof{``Language''}][0.45]    
%       \setcvskillcolumns[\widthof{``Management Tools''}][0.6][3em]%   ajust all at once.
% 
%   Note
%       - For the styles 'classic' and 'casual' the first column is set to hintscolumnwidth such that
%         it aligns with the rest of the entries. A readjustment of the first column should therefor 
%         be avoided. It is recomended to only use \setcvskillcolumns in the form of 
%         \setcvskillcolumns[][<factor>][<width>], thereby leaving the defaults in place for the first column.
% 
    \def\arg@new@hintscolumnwidth{#1}% <-- all these terminal % signs are necessary for the fancy style to not show weird spaces!!!
    \def\arg@new@bodyLengthFactor{#2}% 
    \def\arg@new@experienceWidth{#3}% 
    % Check for empty arguments. Defaults are given. Thus a call of \setcvskillcolumns 
    % without any arguments leads to nonempty arguments \arg@new@hintscolumnwidth and
    % \def\arg@new@bodyLengthFactor{#2}. However, we need to take care of calls like 
    % \setcvskillcolumns[], \setcvskillcolumns[][], \setcvskillcolumns[][][] or even 
    % \setcvskillcolumns[<somelength>][], \setcvskillcolumns[][<somefactor>]  \setcvskillcolumns[][][<length>]
    \ifdefempty{\arg@new@hintscolumnwidth}{%
        % Case \setcvskillcolumns[], \setcvskillcolumns[][] or \setcvskillcolumns[][<somefactor>]
        \ifdefempty{\arg@new@bodyLengthFactor}{%
            % Case \setcvskillcolumns[][] do nothing here and check if third argument is empty
            \ifdefempty{\arg@new@experienceWidth}{%
                % Case \setcvskillcolumns[][][] do nothing here
            }{%
                % Case \setcvskillcolumns[][][<length>]. reset \cvskill@experiencewidth and
                % \cvskill@descriptorwidth accordingly
                \setlength{\cvskill@experiencewidth}{\arg@new@experienceWidth}%
                \setlength{\cvskill@descriptorwidth}{\skillmatrix@columnwidth-\cvskill@width-\cvskill@experiencewidth}%
            }%
        }{%
            % Case \setcvskillcolumns[][<somefactor>], \setcvskillcolumns[][<somefactor>][<possilly length>]
            \setlength{\skillmatrix@columnwidth}{\arg@new@bodyLengthFactor\skillmatrix@bodylength}%
            \ifdefempty{\arg@new@experienceWidth}{%
                % Case \setcvskillcolumns[][<somefactor>][] do nothing here
            }{%
                % Case \setcvskillcolumns[][<somefactor>][<length>]. reset \cvskill@experiencewidth and
                % \cvskill@descriptorwidth accordingly
                \setlength{\cvskill@experiencewidth}{\arg@new@experienceWidth}%
                \setlength{\cvskill@descriptorwidth}{\skillmatrix@columnwidth-\cvskill@width-\cvskill@experiencewidth}%
            }%            
            \setlength{\cvskill@descriptorwidth}{\skillmatrix@columnwidth-\cvskill@width-\cvskill@experiencewidth}%
            \setlength{\skillmatrix@commentwidth}{\skillmatrix@bodylength-\skillmatrix@columnwidth-3\skillmatrix@padding}%
        }%
        % Case \setcvskillcolumns[] nothing needs to be done here recalculate lengths affected by the change
    }{% 
        % Case \setcvskillcolumns, \setcvskillcolumns[<width>], \setcvskillcolumns[<width>][] 
        % or \setcvskillcolumns[<width>][<somefactor>]
        \setlength{\skillmatrix@hintscolumnwidth}{\arg@new@hintscolumnwidth}%
        \setlength{\skillmatrix@bodylength}{\maincolumnwidth-\skillmatrix@hintscolumnwidth-\separatorcolumnwidth}%
        % in case second argument is given but left empty use default
        \ifdefempty{\arg@new@bodyLengthFactor}{%
            % Case \setcvskillcolumns[<width>][] do nothing here and use default
            % \skillmatrix@columnwidth and check third argument
            \ifdefempty{\arg@new@experienceWidth}{%
                % Case \setcvskillcolumns[<width>][][] do nothing here
            }{%
                % Case \setcvskillcolumns[<width>][][<length>]. reset \cvskill@experiencewidth and
                % \cvskill@descriptorwidth accordingly
                \setlength{\cvskill@experiencewidth}{\arg@new@experienceWidth}%
%                 \setlength{\cvskill@descriptorwidth}{\skillmatrix@columnwidth-\cvskill@width-\cvskill@experiencewidth}%
            }%
        }{%
            % Case \setcvskillcolumns, \setcvskillcolumns[<width>], \setcvskillcolumns[<width>][<somefactor>]
            \setlength{\skillmatrix@columnwidth}{\arg@new@bodyLengthFactor\skillmatrix@bodylength}%
            \ifdefempty{\arg@new@experienceWidth}{%
                % Case \setcvskillcolumns[<width>][<somefactor>][] do nothing here
            }{%
                % Case \setcvskillcolumns[<width>][<somefactor>][<length>]. reset \cvskill@experiencewidth and
                % \cvskill@descriptorwidth accordingly
                \setlength{\cvskill@experiencewidth}{\arg@new@experienceWidth}%
%                 \setlength{\cvskill@descriptorwidth}{\skillmatrix@columnwidth-\cvskill@width-\cvskill@experiencewidth}%
            }%
        }%
        \setlength{\cvskill@descriptorwidth}{\skillmatrix@columnwidth-\cvskill@width-\cvskill@experiencewidth}%
        \setlength{\skillmatrix@commentwidth}{\skillmatrix@bodylength-\skillmatrix@columnwidth-3\skillmatrix@padding}%
    }%
}%


% %-------------------------------------------------------------------------------
% %                \cvskillentry 
% %-------------------------------------------------------------------------------
% \cvskillentry[*][<post_padding>]{<skill_cathegory>}{<0-5>}{<skill_name>}{<years_of_experience>}{<comment>}%
\DeclareDocumentCommand\cvskillentry{s +O{} +m +m +m +m +m}{}%
%     add cvskill matrix row.
% 
%     Input:
%         asterix (optional): include horizontal (dashed) line above the entered line. This behaviour depends on the body style. 
%                             For the fancy style, the asterix has no meaning.
%         input_1 (optional): padding length appended to the legend, default: <0.25em>
%         input_2: string, naming skill cathegory, default: <>
%         input_3: integer between 0 and 5, describing level of skill. \cvskill{input_2} is called internally, default: <>
%         input_4: string, naming the skill, default: <>
%         input_5: positive real number, stating the number of years of experience with this skill , default: <>
%         input_6: string, explaining details w.r.t. that particual skill default: <>
% 
%     Example usages:
%         \cvskillentry*{Language:}{3}{Python}{2}{I have done a million projects with Python}
%         \cvskillentry{}{2}{Lilypond}{14}{So much sheet music! Man I'm the best!}
%         \cvskillentry{}{3}{\LaTeX}{14}{Clearly I rock at \LaTeX}
%         \cvskillentry*[1.5em]{OS:}{3}{Linux}{2}{I only use Archlinux}
% 
%     Note:   
%         - The width of the columns can be adjusted by the \setcvskillcolumns command, see \setcvskillcolumns.

% Definition of \cvskillentry
%\RenewDocumentCommand\cvskillentry{s +O{.25em} +m +m +m +m +m}{%
%    %test for the star * in the command
%    \IfBooleanTF{#1}{% If a star is seen a dotted line is drawn above the entry
%        \begingroup
%            \renewcommand{\arraystretch}{1.1}
%            \arrayrulecolor{awesome}
%            % [*][<post_padding>]{<skill_category>}{<0-5>}{<skill_name>}{<years_of_experience>}{<comment>}%
%            \begin{tabular}{| p{\skillmatrix@hintscolumnwidth}@{\hspace{\separatorcolumnwidth}} | p{\cvskill@width}@{\hspace{\skillmatrix@padding}} | p{\cvskill@descriptorwidth}@{\hspace{\skillmatrix@padding}} | p{\cvskill@experiencewidth}@{\hspace{\skillmatrix@padding}} | p{\skillmatrix@commentwidth}@{} | } %
%                \cdashline{2-5}[.6pt/1pt] %
%                \raggedleft \entrypositionstyle{#3} & \centering \cvskill{#4} & \centering {\skilldetailstyle{#5}} & \centering \skilldetailstyle{#6} & {\skilldetailstyle{ #7}}%
%                %\raggedleft \hintstyle{#3} & \centering \cvskill{#4} & \centering {\skilldetailstyle{#5}} & \centering \skilldetailstyle{#6} & {\skilldetailstyle{ #7}}%
%            \end{tabular}%
%        \endgroup
%        \par\addvspace{#2}
%    }{
%% If no star is seen no line is drawn
%\begin{tabular}{|p{\skillmatrix@hintscolumnwidth}@{\hspace{\separatorcolumnwidth}} | %
%p{\cvskill@width}@{\hspace{\skillmatrix@padding}} | %
%p{\cvskill@descriptorwidth}@{\hspace{\skillmatrix@padding}} | %
%p{\cvskill@experiencewidth}@{\hspace{\skillmatrix@padding}} | %
%p{\skillmatrix@commentwidth}@{} |}%
%%\raggedleft \skilltypestyle{#3} & \centering \cvskill{#4} & \centering {\skilldetailstyle{#5}} & \centering \skilldetailstyle{#6} & {\skilldetailstyle{ #7}}%
%\raggedleft \entrypositionstyle{#3} & \centering \cvskill{#4} & \centering {\skilldetailstyle{#5}} & \centering \skilldetailstyle{#6} & {\skilldetailstyle{ #7}}%
%
%\end{tabular}%
%\par\addvspace{#2}
%}
%}


    \DeclareDocumentCommand\@starIndependentMatrixEntry{}{}%
    \RenewDocumentCommand\cvskillentry{s O{.25em} +m +m +m +m +m}{%
        \arrayrulecolor{awesome}%
        \setlength\arrayrulewidth{\separatorrulewidth}%
        \vspace*{-\separatorrulewidth}% HACK; I don't understand where the space is coming from, nor what it's exact value is :(
        %test for the star * in the command. 
        \RenewDocumentCommand{\@starIndependentMatrixEntry}{}{%
        %\fbox{
            \begingroup%
                % \renewcommand{\arraystretch}{1.25}%
%                \addvbuffer[1pt 2pt]
                \begin{tabular}[c!]{ @{} p{\hintscolumnwidth}@{\hspace{\skillmatrix@padding}}%
                                %@{\hspace{\separatorcolumnwidth}}@{\hspace{\separatorcolumnwidth}} 
                                         p{\skillmatrix@hintscolumnwidth} @{\hspace{\skillmatrix@padding}}%
                                         p{\cvskill@width}@{\hspace{\skillmatrix@padding}} | %
                                         p{\cvskill@descriptorwidth}@{\hspace{\skillmatrix@padding}} | %
                                         p{\cvskill@experiencewidth}@{\hspace{\skillmatrix@padding}} p{\skillmatrix@commentwidth}@{}}%
                     %\cline{3-6}%
    %\hline
                    & \entrypositionstyle{#3}\hspace*{3\skillmatrix@padding} & \centering \cvskill{#4} & \centering \skilldetailstyle {#5} &  \centering \skilldetailstyle{#6} & \skilldetailstyle{#7} \\[#2]%                  
                    % \skilldetailstyle{#7} \\[#2]%
                \end{tabular}%
            \endgroup%
            %}
        }%
        \IfBooleanTF{#1}{% the star does not do anything here
            \@starIndependentMatrixEntry%
        }{% 
            \@starIndependentMatrixEntry%
        }%
        \\[-0.5\baselineskip] %\newline\@aftersectionfalse\ignorespaces%
    }%

\makeatother
%%%%%%%%%%%%%%%%

% 1 : skill type
% 2 : columns 2,3,4
% 3 : size of year 
\setcvskillcolumns[\widthof{ \entrypositionstyle{Presentations111}} ][0.425][2em]

% \setcvskillcolumns[][0.3][2em]

\hypersetup{%
  pdftitle={David Sinden - CV},
  pdfauthor={David Sinden},
  pdfsubject={Curriculum Vitae},
  pdfkeywords={CV, Applied Maths, Ultrasound}
}


\begin{document}

% Is there a top level summary?

% Print the header with above personal informations
% Give optional argument to change alignment(C: center, L: left, R: right)
\makecvheader

% Print the footer with 3 arguments(<left>, <center>, <right>)
% Leave any of these blank if they are not needed
% 2019-02-14 Chris Umphlett - add flexibility to the document name in footer, rather than have it be static Curriculum Vitae
\makecvfooter
  {\DTMmonthname{\DTMfetchmonth{mytoday}}~\DTMfetchyear{mytoday}}
    {David Sinden~~~·~~~cv\_sinden}
  {\thepage}


%-------------------------------------------------------------------------------
%	CV/RESUME CONTENT
%	Each section is imported separately, open each file in turn to modify content
%------------------------------------------------------------------------------

\hypertarget{research-positions}{\section{Professional Experience}\label{professional-experience}}

\begin{cventries}
%
\cventry{Senior Research Scientist -- Modelling \& Simulation Group -- Prof. Tobias Preusser}{Fraunhofer Institute for Digital Medicine MEVIS}{Bremen, Germany}{2019 - present}{\begin{cvitems}
\item Thermal ablation simulations for microwave and ultrasound therapies
\item Pharmacokinetic modelling
\item Ultrasound beamforming, transcranial imaging
\end{cvitems}}
%
\cventry{Senior Research Scientist -- Ultrasound \& Underwater Acoustics Group -- Prof. Bajram Zeqiri}{National Physical Laboratory}{Teddington, UK}{2014 - 2019}{\begin{cvitems}
\item Piezo- and pyro-electric sensor modelling using multi-physics finite-element for device design and characterisation
\item Development of computational tools for ultrasound field characterisation
\item Measurement-based simulation for nonlinear propagation through complex media
\end{cvitems}}
%
\cventry{Post-doctoral Research Associate -- Therapeutic Ultrasound Group -- Prof. Gail ter Haar}{Institute of Cancer Research/The Royal Marsden Cancer NHS Foundation Trust}{Sutton, UK}{2011 - 2014}{\begin{cvitems}
\item Design and implementation of treatment planning software for large phased-array ultrasound transducer for transcostal thermal ablation
\end{cvitems}}
%
\cventry{Post-doctoral Research Associate -- Mechanical Engineering -- Prof. Nader Safari | Prof Eleanor Stride}{University College London}{London, UK}{2008 - 2011}
{\begin{cvitems}
\item Modelling cavitation activity in tissue during high-intensity focus ultrasound therapy.
\end{cvitems}}
%
\end{cventries}
%
%
\hypertarget{education}{\section{Education}\label{education}}
%
\begin{cventries}
\cventry{Dynamical Systems}{PhD -- University College London}{}{2004 - 2008}{%
\begin{cvitems}% 
\item \enquote{Intgerability, Localisation and Bifurcation of an Elastic Conducting rod in a Uniform Magnetic Field}
\item Supervisor: Prof. Gert van der Heijden % | Examiners Prof Gabriel Lord \& Prof Darryl Holm
\end{cvitems}}%\vspace{-.0mm}

\cventry{Modern Applications of Mathematics}{MSc -- University of Bath}{}{2003 - 2004}{}%\vspace{-4.0mm}
%
\cventry{Maths with Applied Mathematics \& Theoretical Physics  -- 2.1}{BSc -- Imperial College London}{}{2000 - 2003}{}%\vspace{-4.0mm}
\end{cventries}
%
%
\hypertarget{skills}{\section{Skills}\label{skills}}
%\noindent

% software as highlevel, low level
%
%\begin{mdframed}
% \cvskillentry[*][<post_padding>]{<skill_cathegory>}{<0-5>}{<skill_name>}{<years_of_experience>}{<comment>}
\cvskillentry*{Software:}{3}{Python}{10}{Contributor to open source projects, open sourcing code in papers}
\cvskillentry[-2pt]{}{3}{C++}{8}{Including STL, Boost, Eigen, VTK and ITK libraries}
\cvskillentry[-2pt]{}{2}{Accelerators}{4}{OpenCL, numba/cupy, jax/XLA}
\cvskillentry[-2pt]{}{3}{Matlab/Octave}{20}{Was used extensively in research}
\cvskillentry[-2pt]{}{3}{Fortran}{14}{Experience from MSc, PhD, post-doc (BLAS, Lapack, auto07, numerical integration)}
\cvskillentry*{OS:}{3}{Linux}{14}{Ubuntu/WSL}% notice the use of the starred command and the optional
\cvskillentry*{Presentation:}{3}{-}{10}{html/css (tailwind), \LaTeX, Bib\TeX}% notice the use of the starred command and the optional
\cvskillentry*{Methods:}{3}{Software development}{8}{Version control (git/svn), continuous integration, build systems (CMake, qmake), testing (googletests), documentation (doxygen, sphinx)}

\newpage
\hypertarget{teaching-dissemination}{\section{Teaching \& Dissemination}\label{teaching-dissemination}}

\begin{cventries}
%
\cventry{Adjunct Lecturer}{Department of Mobility - Constructor University Bremen}{Bremen, Germany}{2022, 2024, 2025}{%
\begin{cvitems}
\item Calculus and Linear Algebra for Graduate Students [MDE-MET-01]: introductory source for MSc students.
\item Numerical Methods [JTMS-MAT-13]: second year course for engineering and mathematics students.
\item Numerical Analysis [CA-S-MATH-804]: final year course for mathematics students.
\end{cvitems}}
%
\cventry{Guest Lecturer}{CIMPA Summer School - University of Havana}{Havana, Cuba}{June 2023}{%
\begin{cvitems}
\item Delivered short lecture course on \enquote{Examples and Principles of Mathematical Modelling in Medicine}, to around 50 applied mathematicians from South America and Africa.
\end{cvitems}}
%
\cventry{Seminar Teacher}{Department of Mechanical Engineering - University College London}{London, UK}{2011}{%
\begin{cvitems}
\item Modelling and Analysis in Engineering I [MECH1010]: first year course for engineering students.
\end{cvitems}}
\cventry{Seminar Teacher}{Department of Civil, Environmental \& Geomatic Engineering - University College London}{London, UK}{2008, 2009}{%
\begin{cvitems}
\item Mathematics for Engineers II [Math6502]: Second maths for course for engineering students.
\end{cvitems}}
%
\cventry{Administration | Seminar Teacher}{Nazarbayez University}{Astana, Kazakhstan}{2011}{%
For students on prestigious \enquote{Bolashak} scholarship and in establishing partner campus at Nazarbayez University\newline
%\newline
\begin{itemize}[leftmargin=2ex, nosep, noitemsep, after=0pt]
\setlength{\parskip}{0pt}
\item Involved in construction of mathematics modules for new courses in mechanical and civil engineering degrees, design of syllabus and preparation of course notes
\begin{itemize}
\item[\textcolor{awesome}{$\boldsymbol{-}$}] 
Project management skills, including liaising with host organisation and participating partner institutions from the United States
\end{itemize}
\item Ran tutorials and revision classes for students in mathematics and physics foundation classes
\item Marking of tests and exam scripts
\end{itemize}}
%
\cventry{Constructor University, Bremen}
{MSc Secondary Supervisor}
{}
{2019-2021}{%
\begin{cvitems}
\item Sandeep Gyawali, Dept. Mathematics, (with Prof. Tobias Preusser).\newline{}\enquote{Extending Composite Finite Element Method for PDE Problems with Geometric Uncertainties}.
\end{cvitems}}
%
\cventry{University College London}{PhD Industrial Supervisor}{}{2018--2019}%
{\begin{cvitems}%
\item Morgan Roberts, Dept. Medical Physics (with Prof. Ben Cox) \enquote{Ultrasound Computed Tomography of the Breast}.
\item Santeri Kaupinm\"{a}ki, Dept. Medical Physics (with Prof. Simon Arridge) \enquote{Inverse Problems for Ultrasound Computed Tomography of the Breast}. 
\end{cvitems}}
%
\cventry{Heriot--Watt University}{}{}{2017--2019}%
{\begin{cvitems}%
\item Katherine Baker, Dept. Mathematics, (with Prof. Lehel Banjai) \enquote{Linear and Nonlinear Wave Equation Models with Power Law Attenuation}.
\end{cvitems}}
%
\cventry{National Physical Laboratory}{Undergraduate Supervision}{}{2015}%
{\begin{cvitems}%
\item Antoine Lucquiaud, \'{E}cole Normale Sup\'{e}rieure de Cachan, \enquote{Boundary Element Methods for Bubble Activity}.
\end{cvitems}}
%
\cventry{University College London}{}{}{2012}{%
\begin{cvitems}%
\item Jade Junqua, ENSEIRB-MATMECA and Bordeaux 1, \enquote{Investigating mode conversion and heating around the ribs due to high-intensity ultrasound}.
\end{cvitems}}
%
\cventry{}{Outreach}{}{}{%\vspace*{-0.5cm}
\begin{cvitems}%
\item Scientific consultant on documentary \enquote{The healing power of sound} (2014)
\item NPL \enquote{Scientific Ambassador}: delivered talks at number of schools and colleges on careers in science as well as demonstrations of experiments relating to objective measurements (2016-2019).
\end{cvitems}} % Mentoring
%
\end{cventries}


%\hypertarget{workshops}{\subsection{Workshops}\label{workshops}}
%
%\begin{cventries}
%    \cventry{Instructor}{Monthly internal R-workshops for LCBC}{Center for Lifespan Changes in Brain and Cognition}{Monthly 2018 - present}{\begin{cvitems}
%\item 2 hour workshops in using R for analysis, visualization, dissemination etc.
%\end{cvitems}}
%    \cventry{Instructor}{Workshop: Straightforward introduction to mixed models \href{https://www.meetup.com/rladies-london/events/259655336/}{\faicon{globe}}}{Oslo UseR!}{June 5, 2019}{\begin{cvitems}
%\item A short workshop in the use of Mixed-models for repeated measurement data
%\end{cvitems}}
%    \cventry{Instructor}{Linear Mixed models on repeated measurement data
%\href{https://www.meetup.com/Oslo-useR-Group/events/260303778/}{\faicon{globe}}}{R-Ladies London}{March 28, 2019}{\begin{cvitems}
%\item A short workshop in the use of Mixed-models for repeated measurement data
%\end{cvitems}}
%    \cventry{Co-instructor}{TidyVerse R \href{https://www.ub.uio.no/english/courses-events/courses/other/Carpentry/software-carpentry/time-and-place/180925-26_TidyR}{\faicon{globe}}}{University of Oslo - Software Carpentry}{Sept.  25 - 26, 2018}{\begin{cvitems}
%\item Two-day workshop on using R and the Tidyverse-packages for data handling and analysis
%\end{cvitems}}
%\end{cventries}

%\hypertarget{research-software-development}{\section{Research software development}\label{research-software-development}}
%
%\footnotesize
%
%A recent interest and professional endeavor is creating R-packages to
%improve data workflows and visualization in R. Icons link to package
%websites with documentation (\faicon{globe}), and github repositories
%(\faicon{github}) where source code is openly available.
%
%\begin{cvhonors}
%    \cvhonor{}{\textbf{ggseg \href{https://github.com/LCBC-UiO/ggseg}{\faicon{github}} \href{https://lcbc-uio.github.io/ggseg/}{\faicon{globe}}}: Lead developer \newline Visualization tool for brain atlas segmentations through R}{}{2018 - present}
%    \cvhonor{}{\textbf{ggeg3d \href{https://github.com/LCBC-UiO/ggseg3d}{\faicon{github}} \href{https://lcbc-uio.github.io/ggseg3d/}{\faicon{globe}}}: Lead developer \newline 3 dimensional visualization tool for brain atlas segmentations through R}{}{2018 - present}
%    \cvhonor{}{\textbf{ggegExtra \href{https://github.com/LCBC-UiO/ggsegExtra}{\faicon{github}} \href{https://lcbc-uio.github.io/ggsegExtra/}{\faicon{globe}}}: Lead developer \newline Repository of atlas data for the ggseg-packages}{}{2018 - present}
%    \cvhonor{}{\textbf{nettskjemar \href{https://github.com/LCBC-UiO/nettskjemar}{\faicon{github}} \href{https://lcbc-uio.github.io/nettskjemar/}{\faicon{globe}}}: Lead developer \newline Package to retrieve data and meta-data from the nettskjema questionnaire tool developed by the University of Oslo}{}{2019 - present}
%    \cvhonor{}{\textbf{metagam \href{https://github.com/Lifebrain/metagam}{\faicon{github}} \href{https://lifebrain.github.io/metagam/}{\faicon{globe}}}: Contributor \newline Meta-Analysis of Generalized Additive Models in Neuroimaging Studies}{}{2020}
%\end{cvhonors}

\newpage
\hypertarget{Awards, affiliations, achievements}{\section{Affiliations, Awards \& Achievements}\label{awards}}
%
\begin{cventries}
%
\cventry{}{Standardiation}{}{}{\begin{cvitems}%
\item Member of IEC Technical Committee 87: Ultrasonics, Working Group 6 -- High Power. \newline{}%
Part of working group of internationally recognised experts writing the technical specification \enquote{TS 63900: Measurement-based Simulation in water and complex media}\end{cvitems}}
%
\cventry{}{Scholarships}{}{}{%\vspace*{-0.5cm}
\begin{cvitems}%
\item M.Sc. funded by an EPSRC scholarship (2003--4), EPSRC funding was awarded for Ph.D. (2004--7) and post-doctoral work (2014)
\end{cvitems}}
%
\cventry{}{Awards}{}{}{%\vspace*{-0.5cm}
\begin{cvitems}
\item Challenge Award: Joint first place in IEEE IUS Challenge on Ultrasound Beamforming with Deep Learning (CUBDL) for \enquote{Improving image quality of single plane wave ultrasound via deep learning based channel compounding (2020)}
\item Conference Award: Honourable mention for paper \enquote{Studying the effect of tissue properties on radiofrequency ablation by visual simulation ensemble analysis} VCBM 2022: Eurographics Workshop on Visual Computing for Biology and Medicine (2022)
\end{cvitems}}
%
\cventry{}{Professional Affiliations}{}{}{%\vspace*{-0.5cm}
\begin{cvitems}%
\item Member of the Society for Industrial and Applied Mathematics member (2004--present), associate member of the Institute of Mathematics (2016--present), and member of the Institute of Physics (2008--present)
\end{cvitems}}
%
\cventry{}{Service}{}{}{%\vspace*{-0.5cm}
\begin{cvitems}
\item Reviewer for a number of journals (Int.~J.~Hyperthermia $\cdot$ Ultrasonics $\cdot$ Ultrasound Med. \& Biol. $\cdot$ Med.~Phys. $\cdot$ Comp.~Meth.~Prog.~Biomed. $\cdot$ J. Open Source Softw.), as well as funding agencies (ANR - France, FWF - Austria, Focused Ultrasound Foundation).
\item Mentor to junior staff at NPL (2015--2019)
\item Maintainer in open-source scientific code: \texttt{k-wave-python} \href{https://github.com/waltsims/k-wave-python}{\faicon{github}}, available via \href{https://pypi.org/project/k-Wave-python/}{pypi} \newline{}%
https://doi.org/10.5281/zenodo.10719461
\end{cvitems}}

%
\cventry{}{Equality, Diversity \& Inclusion}{}{}{%\vspace*{-0.5cm} % as one entry is empty
\begin{cvitems}
\item Member of Fraunhofer MEVIS diversity and inclusion task force (2021-)
\item Member of ICR's Athena Swan board (2012)
\end{cvitems}}
%
\end{cventries}

\newpage
\hypertarget{grants}{\section{Grants}\label{grants}}

In descending chronological order:

\begin{cvhonorsStretch}
%
%\hline 
\cvhonorStretch[1.5]%
{Fraunhofer DISCOVER \newline}%
{CompTop: Computational Topology in Medical Imaging}%
{\texteuro{}150,000}%
{2023}
%
%
%\hline 
\cvhonorStretch[1.5]%
{European Metrology Programme for Innovation and Research (EMPIR) \newline}%
{MAIBAI: Developing a Metrological Framework for Assessment of Image-based Artificial Intelligence Systems for Disease Detection}%
{\texteuro{}180,000}%
{2023}
%
\cvhonorStretch[1.5]%
{Fraunhofer-Netzwerk: Simulation \newline}%
{Physics-Informed Neural Networks }%
{\texteuro{}11,000}%
{2022}
%
\cvhonorStretch[1.5]%
{European Metrology Programme for Innovation and Research (EMPIR) \newline}%
{RaCHy: Radiotherapy Coupled with Hyperthermia -- Adapting the Biological Equivalent Dose Concept}%
{\textsterling{}180,000}%
{2019}
%
\cvhonorStretch[1.5]%
{Analysis for Innovators (A4I), with Deltex Medical Devices \newline}%
{Optimizing Oesophageal Doppler Transducers}%
{\textsterling{}26,500}%
{2018}
%
\cvhonorStretch[1.8]%
{Industrial Challenge Strategy Fund, Wave 1, Metrology for Medical Imaging, with Huntleigh Diagnostics \newline}%
{Optimizing Fetal Doppler Transducers}%
{\textsterling{}45,500}%
{2018}
%
\cvhonorStretch[1.9]%
{EPSRC Network+ Therapy Ultrasound Network for Drug Delivery \& Ablation Research (ThUNDDAR) feasibility study \newline}%
{Machine Learning for Cavitation Detection}%
{\textsterling{}26,500}%
{2016}
%
\cvhonorStretch[1.5]%
{NPL Strategic Research Award \newline}%
{Mathematical Modelling of Histotripsy}%
{\textsterling{}25,000}%
{2014}
%
\cvhonorStretch[1.5]%
{EPSRC/ICR Platform Grant \newline}%
{Vascular Remodelling}%
{\textsterling{}25,000}%
{2012}
%
\end{cvhonorsStretch}

% Patent?
%
%\cvlistitem{Focused Ultrasound Foundation invited participant in workshop {\em Pancreatic cancer workshop} fees, expenses, accommodation and flights to Washington D.C., \textsterling{}2,000 (2019)}
%
%\cvlistitem{American Institute of Mathematics invited participant in workshop {\em Nonlinear solvers for high-intensity focused ultrasound} fees, expenses, accommodation and flights to Paolo Alto, California \textsterling{}2,000 (2012)}
%
%\cvlistitem{Institute of Mathematics and its Applications Small Grant \textsterling{}300 (2011)}
%
%\cvlistitem{UCL Graduate School Conference Fund \textsterling{}250 (2010)}
%
%\cvlistitem{UCL Study Assistance Scheme \textsterling{}350 (2008)}
% Ad hoc reviewer:
% Ultrasonics~$\cdot$ Ultrasound Med \& Biol. $\cdot$ Med.~Phys.
% ANR

\newpage
\hypertarget{Book Chapters}{\section{Book Chapters}\label{chapters}}

\begin{cvhonorsLong}
%
\cvhonorStretch[2]{}{\textcolor{awesome}{\underline{David Sinden}}, %
\href{https://doi.org/10.1201/9780429162671-6}{\enquote{Numerical modelling for simulation and planning of focused ultrasound treatments}}\newline In \emph{Image-guided Focused Ultrasound Therapy: Physics and Clinical Applications}, Eds. F.~Wu, G.~ter~Haar, and I.~Rivens, Series in Medical Physics and Biomedical Engineering, (CRC Press, Baton Rogue, FL, 2024) ISBN 9781498711357}{---}{2024}
%
\end{cvhonorsLong}

\hypertarget{publications-preprints}{\section{Publications \& Preprints}\label{publications-preprints}}

\footnotesize
In descending chronological order. Citation data from Google Scholar.

\begin{cvhonorsLong}
%
\cvhonorStretchNew[2]{}{Pauline Coralie Guillemin, \textcolor{awesome}{\underline{David Sinden}}, Yacine M’Rad, Michael Schwenke, Jennifer Le Guevelou, Johan Uiterwijk, Orane Lorton, Max Scheffler, Pierre-Alexandre Poletti, J\"{u}rgen Jenne, Thomas Zilli, and Rares Salomir, %
\href{https://www.mdpi.com/2072-6694/15/1/163/pdf}{\enquote{A novel concept of transperineal focused ultrasound transducer for prostate cancer local deep hyperthermia treatments}}. \emph{Cancers} 15, 163}{cites: 5}{2022}
%\vfill

\cvhonorStretchNew[2]{}{Christina A. Neizert, Hoang N. C. Do, Miriam Zibell, Christian Rieder, \textcolor{awesome}{\underline{David Sinden}}, Stefan M. Niehues, Janis L. Vahldiek, Kai S. Lehmann, and Franz G. M. Poch, \href{https://doi.org/10.1038/s41598-022-21437-4}{\enquote{Three-dimensional assessment of vascular cooling effects on hepatic microwave ablation in a standardized ex vivo model}}, \emph{Sci. Rep.} 12, 17061}{cites: 5}{2022}

%\vfill
\cvhonorStretchNew[1]{}{Karl Heimes, Marina Evers, Tim Gerrits, Sandeep Gyawali, \textcolor{awesome}{\underline{David Sinden}}, Tobias Preusser, and Lars Linsen, \href{https://doi.org/10.2312/vcbm.20221187}{\enquote{Studying the effect of tissue properties on radiofrequency ablation by visual simulation ensemble analysis}}, in Eurographics Workshop on Visual Computing for Biology and Medicine, Eds. R.~G.~Raidou, B.~Sommer, T.~W.~Kuhlen, M.~Krone, T.~Schultz, and H--Y.~Wu (The Eurographics Association, 2022) ISBN 978-3-03868-177-9, ISSN 2070-5786}{cites: 2}{2022}

\cvhonorStretchNew[2]{}{Dongwoon Hyun, Alycen Wiacek, Sobhan Goudarzi, Sven Rothlübbers, Amir Asif, Klaus Eickel, Yonina C. Eldar, Jiaqi Huang, Massimo Mischi, Hassan Rivaz, \textcolor{awesome}{\underline{David Sinden}}, Ruud J. G. van Sloun, Hannah Strohm, and Muyinatu A. Lediju Bell, \href{https://doi.org/10.1109/TUFFC.2021.3094849}{\enquote{Deep learning for ultrasound image formation: CUBDL evaluation framework and open datasets}}, \emph{IEEE Trans. Ultrason. Ferroelectr. Freq. Control} 68, 3466-–3483}{cites: 81}{2021}

\cvhonorStretchNew[2]{}{Santeri Kaupinm\"{a}ki, Ben Cox, Simon Arridge, Christian Baker, \textcolor{awesome}{\underline{David Sinden}}, and Bajram Zeqiri, \href{https://doi.org/10.1088/1361-6501/abc866}{\enquote{Pyroelectric ultrasound sensor model: directional response}}, \emph{Meas. Sci. Technol.} 32, 035106}{cites: 4}{2021}

%\vfill
\cvhonorStretchNew[2]{}{Sven Rothl\"{u}bbers, Hannah Strohm, Klaus Eickel, J\"{u}rgen Jenne, Vincent Kuhlen, \textcolor{awesome}{\underline{David Sinden}}, and Matthias G\"{u}nther, \href{https://doi.org/10.1109/IUS46767.2020.9251322}{\enquote{Improving image quality of single plane wave ultrasound via deep learning based channel compounding}}, \emph{2020 IEEE International Ultrasonics Symposium~(IUS)} pp. 1-–4}{cites: 29}{2020}
%\vfill

\cvhonorStretchNew[2]{}{Nadia A. S. Smith, \textcolor{awesome}{\underline{David Sinden}}, Spencer A. Thomas, Marina Romanchikova, Jessica E. Talbott, and Michael Adeogun, \href{https://doi.org/10.1259/bjr.20190574}{\enquote{Building confidence in digital health through metrology}}, \emph{Br. J. Radiol.} 93, 20190574}{cites: 13}{2020}

%\vfill
\cvhonorStretchNew[2]{}{\textcolor{awesome}{\underline{David Sinden}}, Srinath Rajagopal, N. Christopher Chaggares, Guofeng Pang, and Oleg Ivanytskyy, \href{https://doi.org/10.1109/ULTSYM.2017.8091600}{\enquote{Reducing uncertainties for spatial averaging at high frequencies}}, \emph{2017 IEEE International Ultrasonics Symposium~(IUS)} (IEEE, 2017) pp. 1–4}{cites: 1}{2017}

%\vfill
\cvhonorStretchNew[2]{}{Ki Joo Pahk, Pierre Gélat, \textcolor{awesome}{\underline{David Sinden}}, Dipok Kumar Dhar, and Nader Saffari, \href{https://doi.org/10.1016/j.ultrasmedbio.2017.08.938}{\enquote{Numerical and experimental study of mechanisms involved in boiling histotripsy}}, \emph{Ultrasound Med. Biol.} 43, 2848-–2861}{cites: 41}{2017}

%\vfill
\cvhonorStretchNew[2]{}{\textcolor{awesome}{\underline{David Sinden}} and Gail ter Haar, \href{https://doi.org/10.3978/j.issn.2218-676X.2014.10.02}{\enquote{Dosimetry implications for correct ultrasound dose deposition: uncertainties in descriptors, planning and treatment delivery}}, \emph{Trans. Cancer Res.} 3, 459-–471}{cites: 13}{2014}

%\vfill
\cvhonorStretchNew[2]{}{\textcolor{awesome}{\underline{David Sinden}}, Eleanor Stride, and Nader Saffari, \href{https://doi.org/10.1088/1742-6596/353/1/012008}{\enquote{Approximations for acoustically excited bubble cluster dynamics}}, \emph{J. Phys.: Conf. Ser.}, Vol. 353 (IOP Publishing, 2012) p. 012008}{cites: 3}{2012}

\cvhonorStretchNew[2]{}{\textcolor{awesome}{\underline{David Sinden}} and Gert H. M. van der Heijden, \enquote{The buckling of magneto-strictive Cosserat rods}, in Proc. 7$^\text{th}$ European Nonlinear Dynamics Conference (ENOC 2011), edited by D. Bernardini, G. Rega, and F. Romeo (European Mechanics Society, 2011) p. 4, ISBN 978-88-906234-2-4}{---}{2011}
%\vfill

%\vfill
\cvhonorStretchNew[2]{}{Gert H. M. van der Heijden and \textcolor{awesome}{\underline{David Sinden}}, \enquote{Localisation of a twisted conducting rod in a uniform magnetic field: the Hamiltonian-Hopf-Hopf bifurcation}, in Proc. 7$^\text{th}$ European Nonlinear Dynamics Conference (ENOC 2011), edited by D. Bernardini, G. Rega, and F. Romeo (European Mechanics Society, 2011) p. 4, ISBN 978-88-906234-2-4}{cites: 2}{2011}
% \href{https://www.worldcat.org/title/proceedings-of-the-7th-european-nonlinear-dynamics-conference-enoc-2011-july-24-29-2011-rome-italy/oclc/804881785}{\textcolor{cyan}{\texttt{isbn:9788890623424}} }
%\vfill

\cvhonorStretchNew[2]{}{\textcolor{awesome}{\underline{David Sinden}}, Eleanor Stride, and Nader Saffari, \href{https://doi.org/10.1088/1742-6596/195/1/012008}{\enquote{The effects of nonlinear wave propagation on the stability of inertial cavitation}}, \emph{J. Phys.: Conf. Ser.}, Vol. 195 (IOP Publishing, 2009) p. 012008}{cites: 3}{2009}

%\vfill
\cvhonorStretchNew[2]{}{\textcolor{awesome}{\underline{David Sinden}} and Gert H. M. van der Heijden, \href{https://doi.org/10.1088/1751-8113/42/37/375207}{\enquote{Spatial chaos of an extensible conducting rod in a uniform magnetic field}}, \emph{J. Phys. A: Math. Theor.} 42, 375207}{cites: 10}{2009}

%\vfill
\cvhonorStretchNew[2]{}{\textcolor{awesome}{\underline{David Sinden}} and Gert H. M. van der Heijden, \href{https://doi.org/10.1088/1751-8113/41/4/045207}{\enquote{Integrability of a conducting elastic rod in a magnetic field}}, \newline\emph{J. Phys. A: Math. Theor.} 41, 045207}{cites: 10}{2008}

%\vfill
\end{cvhonorsLong}
%\vfill

\hypertarget{presentations}{\section{Presentations}\label{presentations}}

\footnotesize
In descending chronological order. 

\begin{cvhonorsLong}
%
%\cvhonorStretch{}{Nonlinear Dyanmics of Bubble Activity in Tissue, Dynamical Systems Seminar, 23 January 2025 %
%\href{https://djps.github.io/abstracts/bremen/}{[\textcolor{awesome}{Abstract}]}
%\href{https://djps.github.io/presentations/bremen.pdf}{[\textcolor{awesome}{Presentation}]}
%}{Invited}{2023}

\cvhonorStretch{}{Nonlinear dynamics of microbubbles in tissue, Dynamical Systems and Geometry Seminary, University of Bremen, 23 January 2025 %
\href{https://djps.github.io/abstracts/bremen/}{[\textcolor{awesome}{Abstract}]}
\href{https://djps.github.io/presentations/bremen.pdf}{[\textcolor{awesome}{Presentation}]}
}{Invited}{2025}

\cvhonorStretch{}{Integrability, localisation and bifurcation of an elastic conducting rod in a magnetic field, 7$^{\text{th}}$ Workshop on Dynamical Systems \& Ergodic Theory in Northern Germany, 9 June 2023 %
\href{https://djps.github.io/abstracts/hamburg/}{[\textcolor{awesome}{Abstract}]}
\href{https://djps.github.io/presentations/hamburg.pdf}{[\textcolor{awesome}{Presentation}]}
}{Invited}{2023}

\cvhonorStretch{}{Artificial intelligence in therapeutic ultrasound, 22$^{\text{nd}}$ International Symposium on Therapeutic Ultrasound, Lyon, 17--20 April 2023 %
\href{https://djps.github.io/abstracts/lyon/}{[\textcolor{awesome}{Abstract}]}
%\href{https://djps.github.io/presentations/virtual_asa.pdf}{[\textcolor{awesome}{Presentation}]}
}{Invited}{}
%
\cvhonorStretch{}{Patient-specific modelling of microwave ablation, Society for Thermal Medicine 2022 Annual Meeting,\newline{}1--4 May 2022 %
\href{https://djps.github.io/abstracts/stm/}{[\textcolor{awesome}{Abstract}]}
\href{https://djps.github.io/presentations/stm.pdf}{[\textcolor{awesome}{Presentation}]}
}{}{2022}
%
\cvhonorStretch{}{Factors for validation of measurement-based simulation, ASA 179, ASA Acoustics Virtually Everywhere,\newline{}8 December 2020. \href{https://djps.github.io/abstracts/asa_virtual/}{[\textcolor{awesome}{Abstract}]}\href{https://djps.github.io/presentations/virtual_asa.pdf}{[\textcolor{awesome}{Presentation}]} }{}{2020}
% \href{https://doi.org/10.1121/1.5147005}{\textcolor{cyan}{\texttt{doi:10.1121/1.5147005}}}
%
\cvhonorStretch{}{Machine learning for cavitation detection, British Medical Ultrasound Symposium, 5 December 2018}{}{2018}
%
\cvhonorStretch{}{Acceleration techniques for acoustic holography, British Medical Ultrasound Symposium, 8 December 2016}{}{2016}
%
\cvhonorStretch{}{Computational challenges in high-intensity focused ultrasound, University of Strathclyde, 25 October 2016}{Invited}{}
%
\cvhonorStretch{}{Absorption of ultrasound by tissue: fractional operators and integral equations Maxwell Institute for Applied Analysis, International Centre for Mathematical Sciences, Edinburgh, 7 October 2016}{Invited}{}
%
%
% pyro-electric sensor model.
% \href{https://doi.org/10.1121/1.5137366}
%
\cvhonorStretch{}{Mathematical challenges of high-intensity focused ultrasound, Leslie Comrie Lecture, University of Greenwich,\newline{}11 April 2016}{Invited}{}
%
\cvhonorStretch{}{Wave3D: A parallelised three-dimensional nonlinear acoustic wave propagation solver, Anglo-French Physical Acoustics Conference 15, London, 13--15 January 2016}{}{}
%
\cvhonorStretch{}{Computational challenges in high-intensity focused ultrasound treatment planning, University of Surrey, \newline{}15 December 2015}{Invited}{2015}
%
% \cvhonorStretch{}{Wave3D : A parallelized three-dimensional nonlinear acoustic wave propagation solver}, 15$^{\text{th}}$ International Symposium on Therapeutic Ultrasound, Utretch, Netherlands, 15--18 April 2015}{}{}
%
\cvhonorStretch{}{Computational challenges in high-intensity focused ultrasound treatment planning, 14$^{\text{th}}$ International Symposium on Therapeutic Ultrasound, Las Vegas, Nevada, 2--4 April 2014}{}{2014}
%
\cvhonorStretch{}{Treatment planning of high-intensity focused ultrasound, Medical Modelling Group, University College London,\newline{}30 September 2013}{Invited}{2013}
%
\cvhonorStretch{}{The challenges in boundary element modelling for high-intensity focused ultrasound treatment planning, Boundary Integral Equation Methods for High-Frequency Scattering, University of Reading, 25 May 2012}{Invited}{2012} 
%
\cvhonorStretch{}{The effects of nonlinear wave propagation on thermal ablation high-intensity focused ultrasound, Department of Electrical Engineering, Stanford University, California, 11 April 2012}{Invited}{}
%
\cvhonorStretch{}{The buckling of magneto-strictive Cosserat rods, 7$^{\text{th}}$ European Mechanics Society European Nonlinear Oscillations Conference, Rome, Italy, 24--29 July 2011}{}{2011} %\href{http://www.ucl.ac.uk/~ucesdsi/enoc1_abs.html}{[Abstract]} }\\
%
\cvhonorStretch{}{Localisation of a twisted conducting rod in a uniform magnetic field: the Hamiltonian-Hopf-Hopf bifurcation,\newline{}7$^{\text{th}}$ European Mechanics Society European Nonlinear Oscillations Conference, Rome, Italy, 24--29 July 2011}{}{} % \href{http://www.ucl.ac.uk/~ucesdsi/enoc_abs.html}{[Abstract]} }\\
%
\cvhonorStretch{}{Cavitation in tissue under high-intensity focused ultrasound, SIAM Conference on Applications of Dynamical Systems, Snowbird, Utah, 22--26 May 2011}{}{} % \href{http://www.ucl.ac.uk/~ucesdsi/siam_abs.html}{[Abstract]} }\\
%
\cvhonorStretch{}{The effect of fluid compressibility on multi-bubble cavitation for high-intensity focused ultrasound, \newline{}161$^{\text{st}}$ Meeting of the Acoustical Society of America, Seattle, Washington, 23--27 May 2011}{}{} % \href{http://www.ucl.ac.uk/~ucesdsi/asa11_abs.html}{[Abstract]} }\\
% \href{http://dx.doi.org/10.1121/1.3588557}{\textcolor{cyan}{\texttt{doi:10.1121/1.3588557}} }
%
\cvhonorStretch{}{Modelling cavitation in liver tissue under high-intensity focused ultrasound, British Applied Mathematics Colloquium, University of Birmingham, 11--13 April 2011}{}{} % \href{http://www.ucl.ac.uk/~ucesdsi/bamc2011_abs.html}{[Abstract]} }\\
%
\cvhonorStretch{}{Cavitation in models of wave propagation through tissue under high-intensity focused ultrasound, Anglo-French Physical Acoustics Conference 11, Fr{\'e}jus, France, 19--21 January 2011}{}{} % \href{http://www.ucl.ac.uk/~ucesdsi/afpac11_abs.html}{[Abstract]} } \\
%
\cvhonorStretch{}{The influence of liquid viscosity and compressibility on multi-bubble cavitation, UK Therapeutic Ultrasound Interest Group, University College London, 20 December 2010}{}{2010} % \href{http://www.ucl.ac.uk/~ucesdsi/thugs8_abs.html}{[Abstract]}} \\
%
\cvhonorStretch{}{Multi-bubble interactions, and high-intensity focused ultrasound therapy, 10th International Symposium on Therapeutic Ultrasound, Tokyo, 9--12 June 2010}{}{}% \href{http://www.ucl.ac.uk/~ucesdsi/istu10_abs.html}{[Abstract]} } \\
%
\cvhonorStretch{}{On the stability of interacting bubbles, UK Therapeutic Ultrasound Interest Group, Institute of Cancer Research,\newline{}11 May 2010}{}{} %\href{http://www.ucl.ac.uk/~ucesdsi/thugs7_abs.html}{[Abstract]} }\\
%
\cvhonorStretch{}{Cavitation in high-intensity focused ultrasound treatment, Medical Modelling Group, University College London,\newline{}4 May 2010}{}{} % \href{http://www.ucl.ac.uk/~ucesdsi/mmg_abs.html}{[Abstract]} } \\
%
\cvhonorStretch{}{Integrability, spatially complex localisation and bifurcation of an elastic conducting rod in a uniform magnetic field, London Dynamical Systems Workshop, Imperial College, 29 April 2010}{Invited}{} % \href{http://www.ucl.ac.uk/~ucesdsi/ldsg_abs.html}{[Abstract]} }\\
%
\cvhonorStretch{}{Phase synchronisation and the collective instability oscillating bubble clouds, 159$^{\text{th}}$ Meeting of the Acoustical Society of America, Baltimore, Maryland, 19--23 April 2010. \emph{J.\ Acoust.\ Soc.\ Am}.\ 127(3), 1865--1865}{}{} % \href{http://www.ucl.ac.uk/~ucesdsi/balt_ds_abs.html}{[Abstract]}  }\\ % \href{http://dx.doi.org/10.1121/1.3384478}{\textcolor{cyan}{\texttt{doi:10.1121/1.3384478}}
%
\cvhonorStretch{}{On multi-bubble interactions, Anglo-French Physical Acoustics Conference 10, Kendal, 18--22 January 2010}{}{} % \href{http://www.ucl.ac.uk/~ucesdsi/thugs6_abs.html}{[Abstract]} }\\
%
\cvhonorStretch{}{On multi-bubble interactions, UK Therapeutic Ultrasound Interest Group, University College London,\newline{}11 November 2009}{}{2009} % \href{http://www.ucl.ac.uk/~ucesdsi/thugs6_abs.html}{[Abstract]} }\\ (Also presented at the Anglo-French Physical Acoustics Conference 10, Kendal, 18-22 January 2010.)
%
\cvhonorStretch{}{The effects of viscoelasticity on the stability of inertial cavitation, 9$^\text{th}$ International Symposium on Therapeutic Ultrasound, Aix-en-Province, 23--26 September 2009}{}{} % \href{http://www.ucl.ac.uk/~ucesdsi/istu09_abs.html}{[Abstract]} }\\
%
\cvhonorStretch{}{The effects of nonlinear wave propagation on inertial cavitation, UK Therapeutic Ultrasound Interest Group, University College London, 18 December 2008}{}{2008} % \href{http://www.ucl.ac.uk/~ucesdsi/thugs5_abs.html}{[Abstract]} }\\
%
\cvhonorStretch{}{The effects of nonlinear wave propagation on inertial cavitation, Anglo-French Physical Acoustics Conference 9, Arcachon, 8--10 December 2008}{}{} % \href{http://www.ucl.ac.uk/~ucesdsi/Surrey_abs.html}{[Abstract]} }\\ % (Also presented at the UK Theuraputic Ultrasound Interest Group, UCL, 18 December 2008.)
%
\cvhonorStretch{}{Integrability, spatially complex localisation and bifurcation of an elastic conducting rod in a uniform magnetic field, University of Surrey, 3 October 2008}{Invited}{} % \href{http://www.ucl.ac.uk/~ucesdsi/istu10_abs.html}{[Abstract]} }\\
%
\cvhonorStretch{}{Spatially complex localisation of an elastic conducting rod in a uniform magnetic field, Bifurcations in Dynamical Systems with Applications, University of Bielefeld, 19--21 May 2008}{}{} %\href{http://www.ucl.ac.uk/~ucesdsi/bielefeld_abs.html}{[Abstract]} }\\
%
\cvhonorStretch{}{The integrability of a conducting elastic rod in a magnetic field, British Applied Mathematics Colloquium, Bristol University, 17--19 April 2007}{}{2007} % \href{http://www.ucl.ac.uk/~ucesdsi/bamc_abs.html}{[Abstract]} }\\
%
\end{cvhonorsLong}

\end{document}

%% Mentions:
%
%% \enquote{Phase-insensitive versus phase-sensitive ultrasound absorption tomography in the frequency domain} https://arxiv.org/pdf/2202.07157.pdf

%@article{baker2022numerical,
%  title={Numerical analysis of a wave equation for lossy media obeying a frequency power law},
%  author={Baker, Katherine and Banjai, Lehel},
%  journal={IMA Journal of Numerical Analysis},
%  volume={42},
%  number={3},
%  pages={2083--2117},
%  year={2022},
%  publisher={Oxford University Press},
%  doi={10.1093/imanum/drab028}
%}
%
%@inproceedings{gelat2014hifu,
%  title={HIFU scattering by the ribs: constrained optimisation with a complex surface impedance boundary condition},
%  author={G{\'e}lat, P and Ter Haar, G and Saffari, N},
%  booktitle={Journal of Physics: Conference Series},
%  volume={498},
%  pages={012004},
%  year={2014},
%  organization={IOP Publishing},
%  doi={10.1088/1742-6596/498/1/012004}
%}
%
%@article{gelat2011modelling,
%  title={Modelling of the acoustic field of a multi-element HIFU array scattered by human ribs},
%  author={G{\'e}lat, Pierre and Ter Haar, Gail and Saffari, Nader},
%  journal={Phys. Med. Biol.},
%  volume={56},
%  number={17},
%  pages={5553},
%  year={2011},
%  publisher={IOP Publishing},
%  doi={10.1088/0031-9155/56/17/007}
%}
%
%@article{jensen2013real,
%  title={Real-time temperature estimation and monitoring of HIFU ablation through a combined modeling and passive acoustic mapping approach},
%  author={Jensen, C. R. and Cleveland, R. O. and Coussios, C. C.},
%  journal={Phys. Med. Biol.},
%  volume={58},
%  number={17},
%  pages={5833},
%  year={2013},
%  publisher={IOP Publishing},
%  doi={10.1088/0031-9155/58/17/5833}
%}
%
%@inproceedings{gyongy2011characterization,
%  title={Characterization of cavitation based on autocorrelation of acoustic emissions},
%  author={Gy{\"o}ngy, Mikl{\'o}s and Jensen, Carl},
%  booktitle={Proceedings of Meetings on Acoustics 161 ASA},
%  volume={12},
%  number={1},
%  pages={075001},
%  year={2011},
%  organization={Acoustical Society of America},
%  doi={10.1121/1.4793563}
%}
