I have over 15 years' experience in the development of therapeutic ultrasound systems, having started as a post-doctoral research associate at University College London in 2008, and then at the Institute of Cancer Research on developing a transcostal therapeutic ultrasound system which was used in a proof-of-concept animal study. My role was to produce treatment planning software, and this gave me exposure to requirement on documentation, testing, integration with hardware and safety requirements. 

After this I obtained a permanent position the National Physical Laboratory, where, as member of IEC TC 87 WG 6, I developed standards to assess predictions of acoustic fields from ultrasound devices against established measurands. This was not limited to therapeutic, but also diagnostic devices. 

In parallel to this I helped develop, sona, an ultrasound breast imaging system which has recently come to market. My role was to develop computational models which could ensure that the design could be validated against data and was operating at safe levels. Again, this required close collaboration with engineers.

After this I took a position at a Fraunhofer Institute in Germany, where I have produced computational models used in clinically approved cryoablation device and developed models for microwave ablation devices for commercial companies. As a 13485-accredited institute, quality procedures were followed to ensure safety and regulatory requirements were satisfied. Code was reviewed internally to follow the QMS and audited externally.

As minor points:
* I am a maintainer of open-source code, k-wave-python, which is used by many ultrasound start-ups, sure as Openwater, Nudge and vevex.
* I am involved in a research program, MAIBAI, which aims to help regulators assess the robustness of AI classification tool for mammography screening data, with the aim of leading to support standards on AI deployment to multiple centres.
